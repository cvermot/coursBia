\section{Tableau des unités usuelles en aviation}	
	
	\begin{longtable}{
	|>{\centering}m{1.8cm}
	|>{\centering}m{2.8cm}
	|c
	|>{\centering}m{3.2cm}
	|c
	|}

 \hline
 Grandeur & Unité aéronautique & Abbrev. &  Valeur exacte & Valeur approchée\\
 \hline
 %\endfirsthead
 Hauteur Altitude & pieds \anglais{feet}\footnote{Le pied est utilisé pour l'altitude en aéronautique partout dans le monde sauf en Russie, qui utilise le mètre. Les vélivoles (planeurs) utilisent également le mètre pour l'altitude} & ft & 1~ft = 0,3048~m & 1~m $\approx$ 3~pieds \\
 \hline
 Vitesse verticale & pieds/minute & ft/min & 1000~ft/min = 5,08~m/s & 1000~ft/min $\approx$ 5~m/s \\
 \hline
 Distance & mille nautique \anglais{nautical mile} & NM & 1~NM = 1852~m & \\
 \hline
 Vitesse & nœud \anglais{knot} & kt & 1~kt~=~1~NM/h 1~kt~=~1,852~km/h & \\
 \hline
 
 \caption{Tableau récapitulatif des unités utilisées en aéronautique}
 \end{longtable}