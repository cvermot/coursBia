\section{L'alphabet radio international}
\begin{center}
	\begin{longtable}{|L{0.5cm}|L{3.5cm}||L{0.5cm}|L{3.5cm}||L{0.5cm}|L{3.5cm}|}
		\hline
			A & Alpha & 				J & Juliett &			S & Sierra \tabularnewline
		\hline
			B & Bravo & 				K & Kilo &				T & Tango \tabularnewline
		\hline
			C & Charlie & 			L & Lima &				U & Uniform \tabularnewline
		\hline
			D & Delta & 				M & Mike &				V & Victor \tabularnewline	
		\hline
			E & Echo & 				N & November &			W & Wiskey \tabularnewline
		\hline
			F & Fox-trot\footnote{En France, il est d'usage d'utiliser "Fox" et non "Fox-trot". Cet usage s'explique par le fait que les aéronefs immatriculés en France ont F pour première lettre. "Fox-trot" étant assez long à prononcer, les communauté aéronautique française à peu à peu adopté l'usage de "Fox" pour la lettre F.} 
			& 			O & Oscar &				X & X-Ray \tabularnewline
		\hline
			G & Golf & 				P & Papa &				Y & Yankee \tabularnewline
		\hline
			H & Hotel & 				Q & Quebec &				Z & Zoulou \tabularnewline
		\hline
			I & India & 				R & Romeo &				0 & Zéro \anglais{zero} \tabularnewline
		\hline
			1 & Unité \anglais{one} & 2 & Deux \anglais{two}   & 3 & Trois \anglais{three} \tabularnewline
		\hline
			4 & Quatre \anglais{four} & 5 & Cinq \anglais{five} &	6 & Six \anglais{six} \tabularnewline
		\hline
			7 & Sept \anglais{seven} & 	8 & Huit \anglais{eight} & 9 & Neuf \anglais{nine\footnote{Prononcé 'niner'}} \tabularnewline
		\hline
	\caption{L'alphabet radio international}
	\end{longtable}
\end{center}