\section{Classification des aéronefs}
Un \gls{aéronef} est un appareil capable de s'élever et de se mouvoir au sein de l'atmosphère terrestre. On divise les aéronefs en 2 grandes familles :
\begin{itemize}
	\item les  \gls{aérostat}s, qui sont des appareils plus légers que l'air,
	\item les  \gls{aérodyne}s, qui sont plus lourds que l'air.
\end{itemize}

Dans cette partie, nous étudierons les aéronefs mais également les engins spatiaux. Ceux ci ne peuvent être qualifiés d'aéronefs, car, bien que certains d'entre eux puissent se déplacer dans l'atmosphère, ils peuvent également se mouvoir en dehors de celle-ci.

\subsubsection{Pourquoi classer les aéronefs ?}
Chaque type d'aéronef dispose de propriétés et de contraintes qui lui sont propres. La classification des aéronefs en grand groupes présentant des caractéristiques communes permet de leur associer aisément des notions réglementaires (licence de pilote nécessaire, minima météo, zone de vol autorisées...), techniques (fréquences et mode d'entretien, contraintes de conception) ou administratives (immatriculation, assurance...).

\subsection{Aérostats}
	\subsubsection{Montgolfière}
	La montgolfière est un aérostat gonflé à l'air chaud.
	
	Elle est composé d'un ballon (appelé enveloppe) sous lequel est accroché une nacelle dans laquelle prennent place les passagers et le pilote nommé aérostier. Un bruleur généralement alimenté au gaz permet de chauffer l'air contenu dans le ballon. Le pilote chauffe l'air à l'aide du bruleur pour faire monter la montgolfière. \\
	
	La montgolfière ne dispose d'aucun moyen pour se diriger, elle est entièrement soumise aux vents pour ses déplacements. Cependant, le pilote peut exploiter les variation de sens du vent aux différentes altitudes pour orienter son vol dans une certaine mesure. \\
	
	\histoire{La montgolfière a été le premier aéronef conçu par l'humain. Les frères Montgolfier ont conçu le premier ballon à air chaud et réalisé le premier vol en 1783. La même année, ils font voler des animaux puis Jean-François \textbf{Pilâtre de Rozier} et le Marquis d'Arlandes réalisent le premier vol libre humain.}
	\subsubsection{Ballon à gaz}
	Le ballon à gaz est gonflé avec un gaz plus léger que l'air (hydrogène ou hélium).	
	
	\subsubsection{Dirigeable}
	Le dirigeable ou ballon dirigeable est un ballon à gaz équipés d'une systèmes propulsifs lui permettant de se diriger (aussi bien sur le plan horizontal que vertical). \\
	
	Les premiers dirigeables étaient gonflés à l'hydrogène. Ce gaz est dangereux car très inflammable. Les dirigeables modernes sont désormais gonflés à l'hélium. L'hélium est un gaz sur car ininflammable, mais il est plus cher et plus lourd que l'hydrogène (un ballon à l'hélium nécessitera une enveloppe plus grande qu'un ballon à hydrogène de même capacité).

\subsection{Aérodynes}
	\subsubsection{Avions}
	\subsubsection{Planeurs}
	\subsubsection{Hélicoptères}
	\subsubsection{Parapentes, deltaplanes}

\subsection{Engins spatiaux}
	\subsubsection{Lanceurs}
		\paragraph{Fusées}
		\paragraph{Navettes spatiales}
		
	\subsubsection{Engins spatiaux}
		\paragraph{Satellites}
		\paragraph{Sondes}
		
\subsection{Les ULM}

\subsection{Les drones}