% Définition des couleurs de base
\definecolor{gazFroid}{RGB}{173,216,230}     % Bleu clair
\definecolor{gazChaud}{RGB}{255,0,0}         % Rouge
\definecolor{gazExplosion}{RGB}{255,255,0}   % Jaune
\definecolor{gazBrule}{RGB}{128,128,128}     % Gris

% Définition des couleurs de base
\definecolor{gazFroid}{RGB}{173,216,230}     % Bleu clair
\definecolor{gazChaud}{RGB}{255,0,0}         % Rouge
\definecolor{gazExplosion}{RGB}{255,255,0}   % Jaune
\definecolor{gazBrule}{RGB}{128,128,128}     % Gris

% Définition de la couleur des gaz en fonction de l'angle
\newcommand{\setGasColor}[1]{%
    \def\angle{#1}%
    % De gris à bleu (-90 à -30)
    \ifnum\angle<-30
        \def\gas{gazBrule!\the\numexpr100-(((\angle+90)*100)/60)\relax!gazFroid}%
    \else
        % Bleu constant (-30 à 90)
        \ifnum\angle<90
            \def\gas{gazFroid}%
        \else
            % De bleu vers rouge (90 à 270)
            \ifnum\angle<270
                \def\gas{gazFroid!\the\numexpr100-((\angle-90)*100/180)\relax!gazChaud}%
            \else
                % De rouge vers jaune (270 à 300)
                \ifnum\angle<300
                    \def\gas{gazChaud!\the\numexpr((\angle-270)*100/30)\relax!gazExplosion}%
                \else
                    % De jaune vers gris (300 à 450)
                    \ifnum\angle<450
                        \def\gas{gazExplosion!\the\numexpr100-((\angle-300)*100/150)\relax!gazBrule}%
                    \else
                        % Gris constant (450 à 630)
                        \def\gas{gazBrule}%
                    \fi
                \fi
            \fi
        \fi
    \fi
}

\begin{animateinline}[autoplay,loop,controls]{60}
 \multiframe{360}{i=-90+2}{%
    \begin{tikzpicture}
    \setGasColor{\i}%
      \coordinate (boiteLegende1) at (0,8.5);
      \coordinate (boiteLegende2) at (0,9);
      \node[rectangle,minimum width=2cm] [fit = (boiteLegende1) (boiteLegende2)] (legende) {};    
    
      \engine{-\i}
      
      % INTAKE STROKE
      \ifnum\i>-92 \ifnum\i<-88 
          \node[align=center,font=\Large] at (legende.center) {\textbf{Admission}};
          \valveL{.2}
          \valveR{.1};
      \fi \fi 
      \ifnum\i>-90 \ifnum\i<90 
          \node[align=center,font=\Large] at (legende.center) {\textbf{Admission}};
          \setGasColor{\i}%
          \fill[\gas, opacity=0.5]
              (VLo) to[out=110,in=0] ++ (-1.5,0.6) -- ($(VLo)+(-1.5,1.3)$) to[out=0,in=110] (VLm) to[out=\vl+5,in=\vl-47] cycle;
          \wall
          \valveL{.3}
          \valveR{.1};
      \fi \fi
      
      % COMPRESSION STROKE
      \ifnum\i>90 \ifnum\i<272 
          \node[align=center,font=\Large] at (legende.center) {\textbf{Compression}};
          \valveL{.1}
          \valveR{.1};
      \fi \fi 
      
      % POWER STROKE
      \ifnum\i>270 \ifnum\i<452 
          \node[align=center,font=\Large] at (legende.center) {\textbf{Explosion-détente}};
          \valveL{.1}
          \valveR{.1};
      \fi \fi
      
      % EXHAUST STROKE
      \ifnum\i>452 \ifnum\i<630
          \node[align=center,font=\Large] at (legende.center) {\textbf{Echappement}};
          \fill[\gas, opacity=0.5]
              (VRo) to[out=60,in=-180] ++ (1.5,0.6) -- ($(VRo)+(1.5,1.3)$) to[out=180,in=60] (VRm) to[out=-220+\vr,in=-180+\vr] cycle;
          \wall 
          \valveL{.1}
          \valveR{.3};
      \fi \fi
      
      % IGNITION
      \ifnum\i>268 \ifnum\i<280 
          \draw[very thin,yellow!70!black,fill=yellow,shift={(X)}]
              ( -15:.20) -- ( -30:.40) -- ( -40:.25) -- ( -50:.40) --
              ( -60:.22) -- ( -70:.40) -- ( -80:.20) -- ( -90:.45) --
              (-100:.24) -- (-110:.40) -- (-120:.25) -- (-130:.40) --
              (-140:.20) -- (-150:.45) -- (-165:.20) to[out=40,in=140] cycle; 
      \fi \fi 
    \end{tikzpicture}
  }
\end{animateinline}