%éléments communalisés
\pgfdeclareimage{case}{01-EtudeAeronefs/img/instruments/alt/alt_case.pdf}

\pgfdeclareimage{asiFace}{01-EtudeAeronefs/img/instruments/asi/asi_face.pdf}
\pgfdeclareimage{asiHand}{01-EtudeAeronefs/img/instruments/asi/asi_hand.pdf}
\pgfdeclareimage{asiCase}{01-EtudeAeronefs/img/instruments/asi/asi_case.pdf}

\pgfdeclareimage{altCase}{01-EtudeAeronefs/img/instruments/alt/alt_case.pdf}
\pgfdeclareimage{altFace1}{01-EtudeAeronefs/img/instruments/alt/alt_face_1.pdf}
\pgfdeclareimage{altFace2}{01-EtudeAeronefs/img/instruments/alt/alt_face_2.pdf}
\pgfdeclareimage{altFace3}{01-EtudeAeronefs/img/instruments/alt/alt_face_3.pdf}
\pgfdeclareimage{altHand1}{01-EtudeAeronefs/img/instruments/alt/alt_hand_1.pdf}
\pgfdeclareimage{altHand2}{01-EtudeAeronefs/img/instruments/alt/alt_hand_2.pdf}

\pgfdeclareimage{aiCase}{01-EtudeAeronefs/img/instruments/ai/ai_case.pdf}
\pgfdeclareimage{aiFace}{01-EtudeAeronefs/img/instruments/ai/ai_face.pdf}
\pgfdeclareimage{aiRing}{01-EtudeAeronefs/img/instruments/ai/ai_ring.pdf}
\pgfdeclareimage{aiBack}{01-EtudeAeronefs/img/instruments/ai/ai_back.pdf}

\pgfdeclareimage{hiCase}{01-EtudeAeronefs/img/instruments/hi/hi_case.pdf}
\pgfdeclareimage{hiFace}{01-EtudeAeronefs/img/instruments/hi/hi_face.pdf}

\pgfdeclareimage{tcCase}{01-EtudeAeronefs/img/instruments/tc/tc_case.pdf}
\pgfdeclareimage{tcFace1}{01-EtudeAeronefs/img/instruments/tc/tc_face_1.pdf}
\pgfdeclareimage{tcFace2}{01-EtudeAeronefs/img/instruments/tc/tc_face_2.pdf}
\pgfdeclareimage{tcBall}{01-EtudeAeronefs/img/instruments/tc/tc_ball.pdf}
\pgfdeclareimage{tcBack}{01-EtudeAeronefs/img/instruments/tc/tc_back.pdf}
\pgfdeclareimage{tcMark}{01-EtudeAeronefs/img/instruments/tc/tc_mark.pdf}

\pgfdeclareimage{vsiCase}{01-EtudeAeronefs/img/instruments/vsi/vsi_case.pdf}
\pgfdeclareimage{vsiHand}{01-EtudeAeronefs/img/instruments/vsi/vsi_hand.pdf}
\pgfdeclareimage{vsiFace}{01-EtudeAeronefs/img/instruments/vsi/vsi_face.pdf}

\pgfdeclareimage{ilsCase}{01-EtudeAeronefs/img/instruments/ils/ils_case.pdf}
\pgfdeclareimage{ilsCaseFixed}{01-EtudeAeronefs/img/instruments/ils/ils_case_fixed.pdf}
\pgfdeclareimage{ilsFace}{01-EtudeAeronefs/img/instruments/ils/ils_face.pdf}
\pgfdeclareimage{ilsFlagGs}{01-EtudeAeronefs/img/instruments/ils/ils_flag_gs.pdf}
\pgfdeclareimage{ilsFlagNav}{01-EtudeAeronefs/img/instruments/ils/ils_flag_nav.pdf}
\pgfdeclareimage{ilsHandGs}{01-EtudeAeronefs/img/instruments/ils/ils_hand_gs.pdf}
\pgfdeclareimage{ilsHanvNav}{01-EtudeAeronefs/img/instruments/ils/ils_hand_nav.pdf}

%altimètre
%   -paramètre 1 : altitude en pieds
%   -paramètre 2 : calage en pouces de mercure
\def\alti#1#2{
	\fill[transparent] (0,0) circle (3) ;
	\node[rotate=(#2-28.0)*100] {\pgfbox[center,center]{\pgfuseimage{altFace1}}};
	\node {\pgfbox[center,center]{\pgfuseimage{altFace2}}};
  	\node[rotate=-{(Mod(#1/10,10000))*(36/1000)}] {\pgfbox[center,center]{\pgfuseimage{altFace3}}};
  	\node[rotate=-{(Mod(#1,1000))*(360/1000)}] {\pgfbox[center,center]{\pgfuseimage{altHand2}}};
    	\node[rotate=-{(Mod(#1,10000))*(360/10000)}] {\pgfbox[center,center]{\pgfuseimage{altHand1}}};
    	%\node {\pgfbox[center,center]{\pgfuseimage{altCase}}};
	\node {\pgfbox[center,center]{\pgfuseimage{case}}};
}

%consevateur de cap
%   -paramètre 1 : cap en degrés
\def\conservateurCap#1{
	\fill[transparent] (0,0) circle (3) ;
  	\node[rotate=#1] {\pgfbox[center,center]{\pgfuseimage{hiFace}}};
    	\node {\pgfbox[center,center]{\pgfuseimage{hiCase}}};
}

%ILS
%   -paramètre 1 : QDM en degrés
%   -paramètre 2 : décallage horizontal
%   -paramètre 3 : décallage vertical
\def\ils#1#2#3{
	\fill[transparent] (0,0) circle (3) ;
		\node {\pgfbox[center,center]{\pgfuseimage{ilsCaseFixed}}};
		\node {\pgfbox[center,center]{\pgfuseimage{ilsFlagGs}}};
		\node {\pgfbox[center,center]{\pgfuseimage{ilsFlagNav}}};
		\node[yshift=-#2*32] {\pgfbox[center,center]{\pgfuseimage{ilsHandGs}}};
		\node[xshift=-#2*32] {\pgfbox[center,center]{\pgfuseimage{ilsHanvNav}}};
		\node[rotate=#1] {\pgfbox[center,center]{\pgfuseimage{ilsFace}}};
    	\node {\pgfbox[center,center]{\pgfuseimage{ilsCase}}};
}

%variomètre
%   -paramètre 1 : taux de descente
\def\vario#1{
	\fill[transparent] (0,0) circle (3) ;
  	\node {\pgfbox[center,center]{\pgfuseimage{vsiFace}}};
   	\node[rotate=-(#1/2000)*172] {\pgfbox[center,center]{\pgfuseimage{vsiHand}}};
    	%\node {\pgfbox[center,center]{\pgfuseimage{vsiCase}}};
	\node {\pgfbox[center,center]{\pgfuseimage{case}}};
}

%horizon
%   -paramètre 1 : roulis
%   -parametre 2 : tangage 
\def\horizon#1#2{
	\fill[transparent] (0,0) circle (3) ;
	\node {\pgfbox[center,center]{\pgfuseimage{aiBack}}};
  	\node[rotate=#1,yshift=-#2*1.25] {\pgfbox[center,center]{\pgfuseimage{aiFace}}};
    	\node[rotate=#1] {\pgfbox[center,center]{\pgfuseimage{aiRing}}};
    	\node {\pgfbox[center,center]{\pgfuseimage{aiCase}}};
}

%indicateur de virage
%   -paramètre 1 : taux virage [-1;1] 
%   -parametre 2 : postion bille [-1;1] 
\def\indicateurVirage#1#2{
	\fill[transparent] (0,0) circle (3) ;
  	\node {\pgfbox[center,center]{\pgfuseimage{tcBack}}};
    	\node {\pgfbox[center,center]{\pgfuseimage{tcFace1}}};
    	\node {\pgfbox[center,center]{\pgfuseimage{tcFace2}}};
    	%\node[xshift=#2*35,yshift=#2*4] {\pgfbox[center,center]{\pgfuseimage{tcBall}}};
	\node[rotate around={#2*15:(0,3.7)}] {\pgfbox[center,center]{\pgfuseimage{tcBall}}};
    	\node[rotate=#1*20] {\pgfbox[center,center]{\pgfuseimage{tcMark}}};
    	%\node {\pgfbox[center,center]{\pgfuseimage{tcCase}}};
	\node {\pgfbox[center,center]{\pgfuseimage{case}}};
}

%badin
%   -paramètre 1 : vitesse [0;200] 
\def\badin#1{
	\fill[transparent] (0,0) circle (3) ;
  	\node {\pgfbox[center,center]{\pgfuseimage{asiFace}}};
    	
	%0 kts : 0°
	%40 kts : 36°
	%70 kts : 90°
	%130 kts : 210°
	%160 kts : 264°
	%200 kts : 312°

	%	Vitesse	Position aiguille	Angle/10 kts
	%	0		0,00°	
	%	40		36,00°		NA
	%	50		54,00°		-18°
	%	60		72,00°		-18°
	%	70		90,00°		-18°
	%	80		110,00°		-20°
	%	90		130,00°		-20°
	%	100		150,00°		-20°
	%	110		170,00°		-20°
	%	120		190,00°		-20°
	%	130		210,00°		-20°
	%	140		228,00°		-18°
	%	150		246,00°		-18°
	%	160		264,00°		-18°
	%	200		312,00°		-12°

	%Entre 0 et 40 kts
	\ifnum#1<41
	\node[rotate=-(((#1)/(40))*(36))] {\pgfbox[center,center]{\pgfuseimage{asiHand}}};
	\fi 
	%Entre 40 et 70 kts
	\ifnum#1>40 \ifnum#1<71
	\node[rotate=-(((#1-40)/(70-40))*(90-36))-36] {\pgfbox[center,center]{\pgfuseimage{asiHand}}};
	\fi \fi
	%Entre 71 et 130 kts
	\ifnum#1>70 \ifnum#1<131
	\node[rotate=-(((#1-70)/(130-70))*(210-90))-90] {\pgfbox[center,center]{\pgfuseimage{asiHand}}};
	\fi \fi
	%Entre 130 et 160 kts
	\ifnum#1>130 \ifnum#1<161
	\node[rotate=-(((#1-130)/(160-130))*(264-210))-210] {\pgfbox[center,center]{\pgfuseimage{asiHand}}};
	\fi \fi
	%Entre 160 et 200 kts
	\ifnum#1>160
	\node[rotate=-(((#1-160)/(200-160))*(312-264))-264] {\pgfbox[center,center]{\pgfuseimage{asiHand}}};
	\fi

    	%\node {\pgfbox[center,center]{\pgfuseimage{asiCase}}};
	\node {\pgfbox[center,center]{\pgfuseimage{case}}};
}

% Définir les clés pour les paramètres
\pgfkeys{
    /tdb/.is family,
    /tdb,
    vitesse/.store in = \tdbVitesse,
    altitude/.store in = \tdbAltitude,
    calageAltitude/.store in = \tdbCalageAltitude,
    vz/.store in = \tdbVz,
    cap/.store in = \tdbCap,
    assiette/.store in = \tdbAssiette,
    inclinaison/.store in = \tdbInclinaison,
    derapage/.store in = \tdbDerapage,
    virage/.store in = \tdbVirage,
    afficherT/.store in = \tdbAfficherT,
    vitesse = 0,
    altitude = 0,
    vz = 0,
    cap = 0,
    assiette = 0,
    inclinaison = 0,
    derapage = 0,
    virage = 0,
    calageAltitude = 30,
    afficherT = false,
}

% Nouvelle commande pour dessiner la planche de bord
\newcommand{\dessinerTdB}[1]{
    \pgfkeys{/tdb, #1}

    \begin{tikzpicture}
        % Définir les coordonnées des sommets de la planche de bord
        \coordinate (A) at (-5, 3.5);
        \coordinate (B) at (19, 3.5);
        \coordinate (C) at (19, -10.5);
        \coordinate (D) at (-5, -10.5);

        % Dessiner le polygone avec le coin supérieur gauche arrondi
        \fill[gray, rounded corners=3cm] (D) -- (A) -- (B) -- (C) -- cycle ;

        % Dessiner le polygone rouge si showredpolygon est vrai
        \ifthenelse{\equal{\tdbAfficherT}{true}}{
            \coordinate (T1) at (-3.2, 3.2);
            \coordinate (T2) at (17.2, 3.2);
            \coordinate (T3) at (17.2, -3.5);  
            \coordinate (T4) at (10.4, -3.5);
            \coordinate (T5) at (10.4, -10.2);
            \coordinate (T6) at (3.6, -10.2);
            \coordinate (T7) at (3.6, -3.5); 	
            \coordinate (T8) at (-3.2, -3.5); 
            \draw[line width=5pt, red] (T1) -- (T2) -- (T3) -- (T4) -- (T5) -- (T6) -- (T7) -- (T8) -- cycle  ;
        }{}

        % Placer les instruments
        \begin{scope}[xshift=0cm, yshift=0cm]
            \badin{\tdbVitesse}
        \end{scope}
        
        \begin{scope}[xshift=7cm, yshift=0cm]
            \horizon{\tdbInclinaison}{\tdbAssiette}
        \end{scope}
        
        \begin{scope}[xshift=14cm, yshift=0cm]
            \alti{\tdbAltitude}{\tdbCalageAltitude}
        \end{scope}
        
        \begin{scope}[xshift=0cm, yshift=-7cm]
            \indicateurVirage{\tdbVirage}{\tdbDerapage}
        \end{scope}
        
        \begin{scope}[xshift=7cm, yshift=-7cm]
            \conservateurCap{\tdbCap}
        \end{scope}
        
        \begin{scope}[xshift=14cm, yshift=-7cm]
            \vario{\tdbVz}
        \end{scope}

        %\draw (B) grid (D);    
        
    \end{tikzpicture}
}

