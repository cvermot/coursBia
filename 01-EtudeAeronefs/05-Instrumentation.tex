%éléments communalisés
\pgfdeclareimage{case}{01-EtudeAeronefs/img/instruments/alt/alt_case.pdf}

\pgfdeclareimage{asiFace}{01-EtudeAeronefs/img/instruments/asi/asi_face.pdf}
\pgfdeclareimage{asiHand}{01-EtudeAeronefs/img/instruments/asi/asi_hand.pdf}
\pgfdeclareimage{asiCase}{01-EtudeAeronefs/img/instruments/asi/asi_case.pdf}

\pgfdeclareimage{altCase}{01-EtudeAeronefs/img/instruments/alt/alt_case.pdf}
\pgfdeclareimage{altFace1}{01-EtudeAeronefs/img/instruments/alt/alt_face_1.pdf}
\pgfdeclareimage{altFace2}{01-EtudeAeronefs/img/instruments/alt/alt_face_2.pdf}
\pgfdeclareimage{altFace3}{01-EtudeAeronefs/img/instruments/alt/alt_face_3.pdf}
\pgfdeclareimage{altHand1}{01-EtudeAeronefs/img/instruments/alt/alt_hand_1.pdf}
\pgfdeclareimage{altHand2}{01-EtudeAeronefs/img/instruments/alt/alt_hand_2.pdf}

\pgfdeclareimage{aiCase}{01-EtudeAeronefs/img/instruments/ai/ai_case.pdf}
\pgfdeclareimage{aiFace}{01-EtudeAeronefs/img/instruments/ai/ai_face.pdf}
\pgfdeclareimage{aiRing}{01-EtudeAeronefs/img/instruments/ai/ai_ring.pdf}
\pgfdeclareimage{aiBack}{01-EtudeAeronefs/img/instruments/ai/ai_back.pdf}

\pgfdeclareimage{hiCase}{01-EtudeAeronefs/img/instruments/hi/hi_case.pdf}
\pgfdeclareimage{hiFace}{01-EtudeAeronefs/img/instruments/hi/hi_face.pdf}

\pgfdeclareimage{tcCase}{01-EtudeAeronefs/img/instruments/tc/tc_case.pdf}
\pgfdeclareimage{tcFace1}{01-EtudeAeronefs/img/instruments/tc/tc_face_1.pdf}
\pgfdeclareimage{tcFace2}{01-EtudeAeronefs/img/instruments/tc/tc_face_2.pdf}
\pgfdeclareimage{tcBall}{01-EtudeAeronefs/img/instruments/tc/tc_ball.pdf}
\pgfdeclareimage{tcBack}{01-EtudeAeronefs/img/instruments/tc/tc_back.pdf}
\pgfdeclareimage{tcMark}{01-EtudeAeronefs/img/instruments/tc/tc_mark.pdf}

\pgfdeclareimage{vsiCase}{01-EtudeAeronefs/img/instruments/vsi/vsi_case.pdf}
\pgfdeclareimage{vsiHand}{01-EtudeAeronefs/img/instruments/vsi/vsi_hand.pdf}
\pgfdeclareimage{vsiFace}{01-EtudeAeronefs/img/instruments/vsi/vsi_face.pdf}

\pgfdeclareimage{ilsCase}{01-EtudeAeronefs/img/instruments/ils/ils_case.pdf}
\pgfdeclareimage{ilsCaseFixed}{01-EtudeAeronefs/img/instruments/ils/ils_case_fixed.pdf}
\pgfdeclareimage{ilsFace}{01-EtudeAeronefs/img/instruments/ils/ils_face.pdf}
\pgfdeclareimage{ilsFlagGs}{01-EtudeAeronefs/img/instruments/ils/ils_flag_gs.pdf}
\pgfdeclareimage{ilsFlagNav}{01-EtudeAeronefs/img/instruments/ils/ils_flag_nav.pdf}
\pgfdeclareimage{ilsHandGs}{01-EtudeAeronefs/img/instruments/ils/ils_hand_gs.pdf}
\pgfdeclareimage{ilsHanvNav}{01-EtudeAeronefs/img/instruments/ils/ils_hand_nav.pdf}

%altimètre
%   -paramètre 1 : altitude en pieds
%   -paramètre 2 : calage en pouces de mercure
\def\alti#1#2{
	\fill[transparent] (0,0) circle (3) ;
	\node[rotate=(#2-28.0)*100] {\pgfbox[center,center]{\pgfuseimage{altFace1}}};
	\node {\pgfbox[center,center]{\pgfuseimage{altFace2}}};
  	\node[rotate=-{(Mod(#1/10,10000))*(36/1000)}] {\pgfbox[center,center]{\pgfuseimage{altFace3}}};
  	\node[rotate=-{(Mod(#1,1000))*(360/1000)}] {\pgfbox[center,center]{\pgfuseimage{altHand2}}};
    	\node[rotate=-{(Mod(#1,10000))*(360/10000)}] {\pgfbox[center,center]{\pgfuseimage{altHand1}}};
    	%\node {\pgfbox[center,center]{\pgfuseimage{altCase}}};
	\node {\pgfbox[center,center]{\pgfuseimage{case}}};
}

%consevateur de cap
%   -paramètre 1 : cap en degrés
\def\conservateurCap#1{
	\fill[transparent] (0,0) circle (3) ;
  	\node[rotate=#1] {\pgfbox[center,center]{\pgfuseimage{hiFace}}};
    	\node {\pgfbox[center,center]{\pgfuseimage{hiCase}}};
}

%ILS
%   -paramètre 1 : QDM en degrés
%   -paramètre 2 : décallage horizontal
%   -paramètre 3 : décallage vertical
\pgfkeys{
    /ilsparam/.is family,
    /ilsparam,
    qdm/.store in = \ilsQdm,
    ecartLoc/.store in = \ilsEcartLoc,
    ecartGlide/.store in = \ilsEcartGlide,
    afficherFlagNav/.store in = \ilsAfficherFlagNav,
    afficherFlagGs/.store in = \ilsAfficherFlagGs,
    qdm = 0,
    ecartLoc = 0,
    ecartGlide = 0,
    afficherFlagNav = false,
    afficherFlagGs = false,
}
\newcommand{\ils}[1]{
    \pgfkeys{/ilsparam, #1}
%\def\ils#1#2#3{
	\fill[transparent] (0,0) circle (3) ;
		\node {\pgfbox[center,center]{\pgfuseimage{ilsCaseFixed}}};
		\ifthenelse{\equal{\ilsAfficherFlagGs}{true}}{
			\node {\pgfbox[center,center]{\pgfuseimage{ilsFlagGs}}};
		}
		\ifthenelse{\equal{\ilsAfficherFlagNav}{true}}{
			\node {\pgfbox[center,center]{\pgfuseimage{ilsFlagNav}}};
		}
		\node[yshift=\ilsEcartGlide*31.75] {\pgfbox[center,center]{\pgfuseimage{ilsHandGs}}};
		\node[xshift=\ilsEcartLoc*31.75] {\pgfbox[center,center]{\pgfuseimage{ilsHanvNav}}};
		\node[rotate=\ilsQdm] {\pgfbox[center,center]{\pgfuseimage{ilsFace}}};
    	\node {\pgfbox[center,center]{\pgfuseimage{ilsCase}}};
}

%variomètre
%   -paramètre 1 : taux de descente
\def\vario#1{
	\fill[transparent] (0,0) circle (3) ;
  	\node {\pgfbox[center,center]{\pgfuseimage{vsiFace}}};
   	\node[rotate=-(#1/2000)*172] {\pgfbox[center,center]{\pgfuseimage{vsiHand}}};
    	%\node {\pgfbox[center,center]{\pgfuseimage{vsiCase}}};
	\node {\pgfbox[center,center]{\pgfuseimage{case}}};
}

%horizon
%   -paramètre 1 : roulis
%   -parametre 2 : tangage 
\def\horizon#1#2{
	\fill[transparent] (0,0) circle (3) ;
	\node {\pgfbox[center,center]{\pgfuseimage{aiBack}}};
  	\node[rotate=#1,yshift=-#2*1.25] {\pgfbox[center,center]{\pgfuseimage{aiFace}}};
    	\node[rotate=#1] {\pgfbox[center,center]{\pgfuseimage{aiRing}}};
    	\node {\pgfbox[center,center]{\pgfuseimage{aiCase}}};
}

%indicateur de virage
%   -paramètre 1 : taux virage [-1;1] 
%   -parametre 2 : postion bille [-1;1] 
\def\indicateurVirage#1#2{
	\fill[transparent] (0,0) circle (3) ;
  	\node {\pgfbox[center,center]{\pgfuseimage{tcBack}}};
    	\node {\pgfbox[center,center]{\pgfuseimage{tcFace1}}};
    	\node {\pgfbox[center,center]{\pgfuseimage{tcFace2}}};
    	%\node[xshift=#2*35,yshift=#2*4] {\pgfbox[center,center]{\pgfuseimage{tcBall}}};
	\node[rotate around={#2*15:(0,3.7)}] {\pgfbox[center,center]{\pgfuseimage{tcBall}}};
    	\node[rotate=#1*20] {\pgfbox[center,center]{\pgfuseimage{tcMark}}};
    	%\node {\pgfbox[center,center]{\pgfuseimage{tcCase}}};
	\node {\pgfbox[center,center]{\pgfuseimage{case}}};
}

%badin
%   -paramètre 1 : vitesse [0;200] 
\def\badin#1{
	\fill[transparent] (0,0) circle (3) ;
  	\node {\pgfbox[center,center]{\pgfuseimage{asiFace}}};
    	
	%0 kts : 0°
	%40 kts : 36°
	%70 kts : 90°
	%130 kts : 210°
	%160 kts : 264°
	%200 kts : 312°

	%	Vitesse	Position aiguille	Angle/10 kts
	%	0		0,00°	
	%	40		36,00°		NA
	%	50		54,00°		-18°
	%	60		72,00°		-18°
	%	70		90,00°		-18°
	%	80		110,00°		-20°
	%	90		130,00°		-20°
	%	100		150,00°		-20°
	%	110		170,00°		-20°
	%	120		190,00°		-20°
	%	130		210,00°		-20°
	%	140		228,00°		-18°
	%	150		246,00°		-18°
	%	160		264,00°		-18°
	%	200		312,00°		-12°

	%Entre 0 et 40 kts
	\ifnum#1<41
	\node[rotate=-(((#1)/(40))*(36))] {\pgfbox[center,center]{\pgfuseimage{asiHand}}};
	\fi 
	%Entre 40 et 70 kts
	\ifnum#1>40 \ifnum#1<71
	\node[rotate=-(((#1-40)/(70-40))*(90-36))-36] {\pgfbox[center,center]{\pgfuseimage{asiHand}}};
	\fi \fi
	%Entre 71 et 130 kts
	\ifnum#1>70 \ifnum#1<131
	\node[rotate=-(((#1-70)/(130-70))*(210-90))-90] {\pgfbox[center,center]{\pgfuseimage{asiHand}}};
	\fi \fi
	%Entre 130 et 160 kts
	\ifnum#1>130 \ifnum#1<161
	\node[rotate=-(((#1-130)/(160-130))*(264-210))-210] {\pgfbox[center,center]{\pgfuseimage{asiHand}}};
	\fi \fi
	%Entre 160 et 200 kts
	\ifnum#1>160
	\node[rotate=-(((#1-160)/(200-160))*(312-264))-264] {\pgfbox[center,center]{\pgfuseimage{asiHand}}};
	\fi

    	%\node {\pgfbox[center,center]{\pgfuseimage{asiCase}}};
	\node {\pgfbox[center,center]{\pgfuseimage{case}}};
}

% Définir les clés pour les paramètres
\pgfkeys{
    /tdb/.is family,
    /tdb,
    vitesse/.store in = \tdbVitesse,
    altitude/.store in = \tdbAltitude,
    calageAltitude/.store in = \tdbCalageAltitude,
    vz/.store in = \tdbVz,
    cap/.store in = \tdbCap,
    assiette/.store in = \tdbAssiette,
    inclinaison/.store in = \tdbInclinaison,
    derapage/.store in = \tdbDerapage,
    virage/.store in = \tdbVirage,
    afficherT/.store in = \tdbAfficherT,
    vitesse = 0,
    altitude = 0,
    vz = 0,
    cap = 0,
    assiette = 0,
    inclinaison = 0,
    derapage = 0,
    virage = 0,
    calageAltitude = 30,
    afficherT = false,
}

%Planche de bord
\newcommand{\dessinerTdB}[1]{
    \pgfkeys{/tdb, #1}

    \begin{tikzpicture}
        % Définir les coordonnées des sommets de la planche de bord
        \coordinate (A) at (-5, 3.5);
        \coordinate (B) at (19, 3.5);
        \coordinate (C) at (19, -10.5);
        \coordinate (D) at (-5, -10.5);

        % Dessiner le polygone avec le coin supérieur gauche arrondi
        \fill[gray, rounded corners=3cm] (D) -- (A) -- (B) -- (C) -- cycle ;

        % Dessiner le polygone rouge si showredpolygon est vrai
        \ifthenelse{\equal{\tdbAfficherT}{true}}{
            \coordinate (T1) at (-3.2, 3.2);
            \coordinate (T2) at (17.2, 3.2);
            \coordinate (T3) at (17.2, -3.5);  
            \coordinate (T4) at (10.4, -3.5);
            \coordinate (T5) at (10.4, -10.2);
            \coordinate (T6) at (3.6, -10.2);
            \coordinate (T7) at (3.6, -3.5); 	
            \coordinate (T8) at (-3.2, -3.5); 
            \draw[line width=5pt, red] (T1) -- (T2) -- (T3) -- (T4) -- (T5) -- (T6) -- (T7) -- (T8) -- cycle  ;
        }{}

        % Placer les instruments
        \begin{scope}[xshift=0cm, yshift=0cm]
            \badin{\tdbVitesse}
        \end{scope}
        
        \begin{scope}[xshift=7cm, yshift=0cm]
            \horizon{\tdbInclinaison}{\tdbAssiette}
        \end{scope}
        
        \begin{scope}[xshift=14cm, yshift=0cm]
            \alti{\tdbAltitude}{\tdbCalageAltitude}
        \end{scope}
        
        \begin{scope}[xshift=0cm, yshift=-7cm]
            \indicateurVirage{\tdbVirage}{\tdbDerapage}
        \end{scope}
        
        \begin{scope}[xshift=7cm, yshift=-7cm]
            \conservateurCap{\tdbCap}
        \end{scope}
        
        \begin{scope}[xshift=14cm, yshift=-7cm]
            \vario{\tdbVz}
        \end{scope}

        %\draw (B) grid (D);    
        
    \end{tikzpicture}
}



\section{L'instrumentation de bord}
	Les instruments de bord permettent à l'équipage de contrôler l'ensemble du vol. On peut les regrouper en 3 grandes familles :
	\begin{itemize}
		\item \textbf{instruments de contrôle du vol} : ils permettent à l'équipage de contrôler les paramètres de base du vol : vitesses horizontale et verticale, attitude et altitude
		\item \textbf{instruments de contrôle de l'appareil} : ils permettent à l'équipage de contrôler les systèmes de l'avion : compte-tour moteur, quantités de carburant restante et consommée, températures...
		\item \textbf{instruments de navigation} : ils permettent à l'équipage de situer la position de l'avion dans l'espace : boussole, compas, GPS, VOR, montre...
	\end{itemize}
	
	\subsection{Les instruments de contrôle primaire du vol}
	\subsubsection{L'indicateur de vitesse}
	 L'\gls{anémomètre}, également appelé \gls{badin} est un instrument qui permet de mesurer la vitesse de l'avion par rapport à l'air. C'est l'un des instruments le plus important dans un avion. 

	\begin{figure}[H]	
	\centering
	\begin{tikzpicture}
		\badin{110}
	\end{tikzpicture}
	\legende{Un "badin"}{tikz::instrumentsBase}
	\end{figure}
	
	\histoire{L'anémomètre qui équipe nos avions a été inventé en 1911 par l'ingénieur français Raoul Badin, qui à donc donné son nom à cet instrument.}
	
	L'anémomètre comporte généralement au moins 3 arcs :
	\begin{itemize}
		\item un arc \textbf{vert} qui représente la vitesse d'utilisation normale de l'avion. Cet arc s'étend de la vitesse de décrochage en lisse à la \acrshort{vno} (\acrlong{vno} - vitesse d'utilisation normale), vitesse à ne pas dépasser en air turbulent
		\item un arc \textbf{blanc} qui représente la vitesse d'utilisation de l'avion volets sortis. Cet arc s'étend de la vitesse de décrochage pleins volets à la \acrshort{vfe} (\acrlong{vfe} - vitesse volets sortis), vitesse à ne pas dépasser lorsque les volets sont sortis
		\item un arc \textbf{jaune} qui représente les vitesses auxquelles il est possible de voler uniquement en air calme. Cet arc s'étend de la \acrshort{vno} à la \acrshort{vne} (\acrlong{vne} - vitesse à ne jamais dépasser). Cet arc est généralement terminé par un trait rouge qui rappelle la \acrshort{vne}.
	\end{itemize}
	
	Sur les avions à trains rentrants, une quatrième vitesse est souvent représentée sur le badin : la \acrshort{vle} (\acrlong{vle} - vitesse maximale trains sortis).
	
	Dans la plupart des avions, le badin est gradué en nœuds. La vitesse limite de chaque extrémité des arcs est bien sur propre à chaque type d'avion.
	
	\alert{Un "badin" indique \textbf{toujours une vitesse par rapport à la masse d'air} dans laquelle évolue l'avion. Il ne faut pas confondre cette vitesse avec la vitesse sol que l'on peut obtenir grâce au GPS par exemple.}
	
	\paragraph{Le tube Pitot}
	
	L'anémomètre fonctionne grâce à un capteur appelé \gls{pitot} situé à l'extérieur de l'appareil. Sur les avions légers, ce capteur est habituellement situé sous l'aile. Sur les avions de ligne, ce capteur est généralement situé sur la partie avant du fuselage. Ces positionnements permettent d'éloigner le Pitot des influences aérodynamiques de surfaces portantes.
	
	Le tube Pitot est une sonde très importante pour le pilotage des aéronefs. C'est pourquoi ce capteur est généralement protégé lorsque l'aéronef est au sol, en mettant dessus un étui communément appelé \textit{flamme}. Lors de la pré-vol, on retire l'étui et on vérifie l'absence d'obstruction. Sur beaucoup d'avion, le tube Pitot peut également être chauffé pour empêcher la formation de glace nuisible à son fonctionnement.
	
	\histoire{Le tube Pitot a été inventé en 1732 par le physicien français Henri Pitot.}
	
	
	\begin{figure}[H]
	\begin{minipage}[c]{0.5\linewidth}
	\includegraphics[width=\linewidth]{01-EtudeAeronefs/img/tubePitotC172.jpg}
	\legende{Tube Pitot équipé de sa protection sur Cessna 172}{img:tubePitotC172}
	\end{minipage}
	\hfill
	\begin{minipage}[c]{0.5\linewidth}
	\includegraphics[width=\linewidth]{01-EtudeAeronefs/img/tubesPitotG6000.jpg}
	\legende{Tubes Pitot sur Bombardier Global 6000}{img:tubesPitotG6000}
	\end{minipage}
	\end{figure}
	
	\subsubsection{L'altimètre}
	
	L'\gls{altimètre} est un instrument qui permet de mesurer l'altitude ou la hauteur par rapport à une référence (par exemple le niveau de la mer ou la hauteur de l'aérodrome). Il fonctionne en mesurant la pression atmosphérique. En effet, la pression atmosphérique diminue avec l'altitude. Mesurer cette pression permet donc de mesurer indirectement l'altitude. \\
	
	\begin{figure}[H]	
	\centering
	\begin{tikzpicture}
		\alti{4500}{29.92}
	\end{tikzpicture}
	\legende{Un altimètre}{tikz::instrumentsBase}
	\end{figure}
	
	\paragraph{Fonctionnement} La pièce qui effectue la mesure dans un altimètre est la \gls{anéroïde}. Il s'agit d'une capsule métallique étanche et conçue pour pouvoir être déformée par la pression atmosphérique. La mesure de cette déformation est ensuite amplifiée pour être affichée sur le cadran de l'altimètre.
	
	\paragraph{Calage altimétrique}
	On a vu que l'altimètre mesure l'altitude en mesurant une pression. La valeur de cette pression varie selon le lieu et le temps. 
	
	L'altimètre est donc équipé d'un bouton rotatif qui permet à l'équipage de modifier la pression de référence. Cette référence est affichée dans une petite fenêtre de l'instrument. La pression de référence est soit communiquée par le contrôle aérien, soit déterminée au sol par le pilote à un point de référence (typiquement : altitude de l'aérodrome).
	
	\info{La pression atmosphérique est mesurée en hectopascal (hPa) en Europe et dans la plupart des pays. Elle est mesurée en pouces de mercure (inHg) aux États-Unis d'Amérique (pression moyenne au niveau de la mer : 1013,25~hPa = 29,92~inHg).}
	
	Il existe 3 principales références en altimétrie :
	\begin{itemize}
		\item le \acrshort{qnh}
		\item le \acrshort{qfe}
		\item le calage standard 1013,25~hPa
	\end{itemize}
	
	\subparagraph{Le calage au QNH}
	\subparagraph{Le calage au QFE}
	\subparagraph{Le calage à la pression standard}
	
	\subsubsection{Le variomètre}
	Le variomètre \anglais{variometer} est un instrument qui indique la vitesse verticale d'un aéronef.
	
	\begin{figure}[H]	
	\centering
	\begin{tikzpicture}
		\vario{-500}
	\end{tikzpicture}
	\legende{Un variomètre}{tikz::instrumentsBase}
	\end{figure}
	
	Le variomètre mesure les variations de pression atmosphériques. Pour cela, il est équipé d'une capsule analogue à celle que l'on trouve dans un altimètre. Cette capsule est reliée à la sonde de pression statique. Cet ensemble est scellé dans un compartiment équipé d'un dispositif de fuite calibré. 
	
	\info{Le variomètre ne peut afficher une vitesse réellement instantanée : il y a un léger délais entre la variation de vitesse verticale et son affichage sur le variomètre.}
	
	\subsubsection{L'horizon artificiel}
	
	\begin{figure}[H]	
	\centering
	\begin{tikzpicture}
		\horizon{0}{0}
	\end{tikzpicture}
	\legende{Un horizon artificiel}{tikz::instrumentsBase}
	\end{figure}
	
	\subsubsection{L'indicateur de virage}
	
	\begin{figure}[H]	
	\centering
	\begin{tikzpicture}
		\indicateurVirage{0}{0}
	\end{tikzpicture}
	\legende{Un indicateur de virage}{tikz::instrumentsBase}
	\end{figure}
	
	\subsection{Les instruments de navigation}
	\subsubsection{Le conservateur de cap}
	
	\begin{figure}[H]	
	\centering
	\begin{tikzpicture}
		\conservateurCap{83}
	\end{tikzpicture}
	\legende{Un conservateur de cap}{tikz::instrumentsBase}
	\end{figure}
	
	
	\subsection{Les instruments de contrôle de la machine}
	
	\subsection{Le glass cockpit}
	
	De plus en plus d'avions sont équipés d'avionique dite "\gls{glasscockpit}". Ce type d'avionique vise à regrouper l'ensemble des instrument sur un ou plusieurs écrans qui occupent alors la majeure partie de l'espace de la planche de bord.
	
	\begin{figure}[H]
	\centering
	\includegraphics[width=0.8\linewidth]{01-EtudeAeronefs/img/G1000-PFD.png}
	\legende{Copie d'écran d'un écran de navigation primaire Garmin G1000}{screenshot::G1000-PFD}
	\end{figure}
	
	\begin{figure}[H]
	\centering
	\includegraphics[width=0.8\linewidth]{01-EtudeAeronefs/img/G1000-MFD.png}
	\legende{Copie d'écran d'un écran multifonction Garmin G1000}{screenshot::G1000-MFD}
	\end{figure}
	
	\info{Les systèmes \gls{glasscockpit} sont désignés par l'acronyme \acrshort{efis} (système d'instruments de vol électronique \anglais{\acrlong{efis}}). Le pilote doit disposer de la qualification EFIS pour piloter un avion glass cockpit.}
	
	\subsection{Conception des planches de bord}