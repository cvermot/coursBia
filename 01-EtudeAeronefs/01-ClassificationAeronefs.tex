%\usetikzlibrary {backgrounds,mindmap}
%\begin{tikzpicture}
%  [root concept/.append style={concept color=blue!80,minimum size=2cm},
%   level 1 concept/.append style={sibling angle=180},
%   level 2 concept/.append style={sibling angle=120},
%   level 3 concept/.append style={sibling angle=90},
%   mindmap]
%  \node [concept] (aeronef) {Aéronef}
%    [clockwise from=0]
%    child[concept color=red] { node[concept] (aerodyne) {Aérodyne}
%    		[clockwise from=90]
%      child[concept color=green] { node[concept] {Voilure fixe} 
%      	[clockwise from=-180]
%      	child[concept color=yellow] { node[concept] {Avion} }
%      	child[concept color=yellow] { node[concept] {Planeur} }
%      	child[concept color=yellow] { node[concept] {Deltaplane} }
%      }
%      child[concept color=green] { node[concept] {Voilure tournante} 
%      	[clockwise from=45]
%      	child[concept color=yellow] { node[concept] {Hélico} }
%      	child[concept color=yellow] { node[concept] {Autogyre} }
%      }
%      child[concept color=green] { node[concept] {Deltaplane} }	
%      }
%    child[concept color=red] { node[concept] (aerostat) {Aérostat}}
%    ;
%\end{tikzpicture}

\section{Classification des aéronefs}
Un \gls{aéronef} \anglais{aircraft} est un appareil capable de s'élever et de se mouvoir au sein de l'atmosphère terrestre. On divise les aéronefs en 2 grandes familles :
\begin{itemize}
	\item les  \gls{aérostat}s \anglais{aerostat/lighter-than-air aircraft}, qui sont des appareils plus légers que l'air,
	\item les  \gls{aérodyne}s \anglais{heavier-than-air aircraft}, qui sont plus lourds que l'air.
\end{itemize}

Dans cette partie, nous étudierons les aéronefs mais également les engins spatiaux. Ceux ci ne peuvent être qualifiés d'aéronefs, car, bien que certains d'entre eux puissent se déplacer dans l'atmosphère, ils peuvent également se mouvoir en dehors de celle-ci.

\subsubsection{Pourquoi classer les aéronefs ?}
Chaque type d'aéronef dispose de propriétés et de contraintes qui lui sont propres. La classification des aéronefs en grand groupes présentant des caractéristiques communes permet de leur associer aisément des notions réglementaires (licence de pilote nécessaire, minima météo, zone de vol autorisées...), techniques (fréquences et mode d'entretien, contraintes de conception) ou administratives (immatriculation, assurance...).

\subsection{Aérostats}
	\subsubsection{Montgolfière}
	La montgolfière \anglais{hot air balloon} est un aérostat gonflé à l'air chaud. \\
	
	Elle est composé d'un ballon (appelé enveloppe) sous lequel est accroché une nacelle dans laquelle prennent place les passagers et le pilote nommé aérostier. Un bruleur généralement alimenté au gaz permet de chauffer l'air contenu dans le ballon. Le pilote chauffe l'air à l'aide du bruleur pour faire monter la montgolfière. \\
	
	La montgolfière ne dispose d'aucun moyen pour se diriger, elle est entièrement soumise aux vents pour ses déplacements. Cependant, le pilote peut exploiter les variation de sens du vent aux différentes altitudes pour orienter son vol dans une certaine mesure.
	
	\begin{figure}[H]
  	\centering
    \includegraphics[width=0.4\textwidth]{01-EtudeAeronefs/img/montgolfiere.jpg}
  	\legende{2 ballons}{img:montgolfiere}
	\end{figure}	
	
	\histoire{La montgolfière a été le premier aéronef habitable conçu par l'humain. Les frères Montgolfier ont conçu le premier ballon à air chaud et réalisé le premier vol en 1783. La même année, ils font voler des animaux (un coq, un mouton et un canard) puis Jean-François \textbf{Pilâtre de Rozier} et le Marquis d'Arlandes réalisent le premier vol libre humain. \\ \\ En 1785, Jean-Pierre Blanchard effectue la première traversée de la Manche avec un ballon, 124 ans avant celle effectuée par Louis Blériot en avion.}
	
	\subsubsection{Ballon à gaz}
	Le ballon à gaz \anglais{gas balloon} est gonflé avec un gaz plus léger que l'air (hydrogène ou hélium).	
	
	\begin{figure}[H]
  	\centering
    \includegraphics[width=0.4\textwidth]{01-EtudeAeronefs/img/ballonAGaz.jpg}
  	\legende{Un ballon à l'hélium moderne}{img:ballonAGaz}
	\end{figure}	
	
	Par rapport à la montgolfière, ce type de ballon permet une autonomie de vol bien plus importante. En effet, il n'y a pas besoin de combustible pour réchauffer le gaz contenu dans le ballon. Ce type de ballon peut donc en théorie reste en l'air sans limite de temps.
	
	\histoire{Le physicien français Jean Charles fera voler le premier ballon à gaz en 1783, la même année que les frères Montgolfier. Toutefois, le premier vol habité en ballon à gaz aura bien lieu après celui des frères Montgolfier.}
	
	Les ballons météo (ballons sonde) sont de type ballons à gaz et sont gonflés à l'hélium.
	
	\subsubsection{Dirigeable}
	Le dirigeable \anglais{airship} ou ballon dirigeable est un ballon à gaz équipé de systèmes propulsifs lui permettant de se diriger (aussi bien sur le plan horizontal que vertical). \\
	
	Les premiers dirigeables étaient gonflés à l'hydrogène. Ce gaz est dangereux car très inflammable. Les dirigeables modernes sont désormais gonflés à l'hélium. L'hélium est un gaz sûr car ininflammable, mais il est plus cher et plus lourd que l'hydrogène (un ballon à l'hélium nécessitera une enveloppe plus grande qu'un ballon à hydrogène de même capacité).
	
	\info{Il existe également des dirigeables à air chaud.}
	
	\begin{figure}[H]
  	\centering
    \includegraphics[width=0.4\textwidth]{01-EtudeAeronefs/img/dirigeable.jpg}
  	\legende{Un dirigeable moderne}{img:dirigeable}
	\end{figure}	
	
	\histoire{Bien que l'idée du ballon dirigeable soit apparue pratiquement simultanément à l'invention des ballons, le premier vol d'un dirigeable eu lieu 1852. On le doit, une fois de plus, à un ingénieur français : Henri Giffard}.

\subsection{Aérodynes}
	\subsubsection{Aéronef à voilure fixe}
		Les appareils à voilure fixe \anglais{fixed-wing aircraft}, également appelés aéroplanes sont des aéronefs dont la sustentation est assurée grâce à des phénomènes aérodynamiques sur des surfaces fixes (ailes). Ce type d'aéronef ne peut se maintenir en l'air que si un flux d'air suffisant existe sur ses surfaces portantes.

		\paragraph{Avions}
		La mise en mouvement de l'avion \anglais{airplane} est assuré par son ou ses moteurs.\\
		
		Il existe une très grande diversité de modèles d'avions, variables en taille, vitesse, conception et usages.
		
		\begin{center}
		\begin{minipage}[c]{1.0\linewidth}
		\begin{figure}[H]
		\begin{minipage}[c]{0.5\linewidth}
		\centering
		\includegraphics[width=0.95\linewidth]{01-EtudeAeronefs/img/CriCri.png}
		\legende{Le CriCri, un des plus petit avion du monde. Ici présenté dans une version à motorisation électrique}{img:CriCri}
		\end{minipage}
		\begin{minipage}[c]{0.5\linewidth}
		\centering
		\includegraphics[width=0.95\linewidth]{01-EtudeAeronefs/img/a380.jpg}
		\legende{L'A380, un des plus gros avion du monde}{img:A380}
		\end{minipage}
		\end{figure}
		\end{minipage}
		\end{center}
		
		\begin{center}
		\begin{minipage}[c]{1.0\linewidth}
		\begin{figure}[H]
		\begin{minipage}[c]{0.5\linewidth}
		\centering
		\includegraphics[width=0.95\linewidth]{01-EtudeAeronefs/img/Rafale.jpg}
		\legende{Un Rafale, avion de chasse}{img:rafale}
		\end{minipage}
		\begin{minipage}[c]{0.5\linewidth}
		\centering
		\includegraphics[width=0.95\linewidth]{01-EtudeAeronefs/img/concorde.jpg}
		\legende{Le Concorde, avion de ligne supersonique}{img:concorde}
		\end{minipage}
		\end{figure}
		\end{minipage}
		\end{center}
		
		\histoire{En France, Clément Ader aurait fait voler son Éole dès 1890, mais il n'existe pas de preuve formelle que cet appareil ait effectivement quitté le sol. \\ \\  Le premier vol attesté d'un aérodyne à moteur a été réalisé par les frères Wright en 1903 aux États-Unis. C'est cette date qui est officiellement retenue comme celle du premier vol d'un avion.}
	
		\paragraph{Planeurs}
		Le planeur \anglais{glider} est un aérodyne dépourvu de moyens de propulsion. Il est donc dépendant de moyens annexes (remorqueur, treuil) pour sa mise en l'air. 
		
	\begin{figure}[H]
  	\centering
    \includegraphics[width=0.4\textwidth]{01-EtudeAeronefs/img/planeur.jpg}
  	\legende{Un planeur}{img:planeur}
	\end{figure}	
		
		Une fois en l'air, le planeur peut exploiter des phénomènes atmosphériques pour gagner de l'altitude. \\
		
		\info{Il existe également des modèles de planeurs équipés de moteurs. Appelés motoplaneurs, ces planeurs peuvent généralement décoller de façon autonome. Le moteur peut également être démarré en vol pour regagner de l'altitude.}
		
		\histoire{Le premier aérodyne dont le vol est formellement attesté est un planeur. Ce vol historique a été réalisé en 1891 par l'ingénieur allemand \textbf{Otto Lilienthal}.}
		
	\subsubsection{Aéronef à voilure tournante}
	Les appareils à voilure tournante \anglais{rotary-wing aircraft}, également appelés aérogires, sont des aéronefs dont la sustentation est assurée par la rotation d'un ou plusieurs rotors.
	
		\paragraph{Hélicoptères}
		L'hélicoptère \anglais{helicopter} est un aérodyne dont la sustentation est assurée exclusivement grâce à un rotor entrainé par un moteur.
		
		Les inventeurs qui ont cherché à concevoir les premières hélicoptères au début du \siecle{20} souhaitaient une machine dont l'usage serait plus souple que les appareils à voilure fixe, en permettant un décollage court voir vertical. \\
		
		\histoire{L'idée de la machine à voilure tournante n'est pas récente, puisque l'on dispose de dessins du Léonard de Vinci datés du \siecle{15} représentant des machines à voilures tournantes.}
		
	\begin{figure}[H]
  	\centering
    \includegraphics[width=0.4\textwidth]{01-EtudeAeronefs/img/helicoptereEC145.jpg}
  	\legende{Un hélicoptère EC145}{img:helicoptereEC145}
	\end{figure}	
		
		De part leur capacité à évoluer en stationnaire, les hélicoptères sont devenus indispensable pour de nombreuses missions : secours, héliportage, transport, épandage, lutte aérienne contre les incendies de foret, travaux sur des lignes électriques, prise de vues...\\
		
		Les principales contreparties à cette souplesse d'usage sont une consommation énergétique supérieure (comparée à un avion) et une vitesse de croisière plus réduite (notamment limitée par le fait que les pales de l'hélicoptère ne doivent pas franchir le mur du son).
		
		\exemple{Le rotor de l'EC145 en illustration ci-dessus mesure 11 m de diamètre et tourne à 383 tours/minutes. Le bout des pales en rotation évolue donc à 795 km/h. La vitesse maximale de l'hélicoptère est de 278 km/h. La pale avançante évolue donc à plus de 1070 km/h lorsque l'hélicoptère vole à sa vitesse maximale.\footnote{Certificat de type de l'EC145/BK117 : https://www.easa.europa.eu/lt/downloads/7941/en}}
		
		Les hélicoptères conçus de façon traditionnelle (un seul rotor horizontal servant à la sustentation et directement entrainé par la force du moteur) sont toujours équipés d'un système anticouple. Il s'agit généralement d'un second rotor, positionné verticalement, appelé  \acrlong{rac} (\acrshort{rac}). En effet, le couple moteur transmis au rotor de sustentation provoque naturellement une rotation en sens inverse de la cellule de l'hélicoptère. Le rotor anticouple contre ce mouvement.
		
		\paragraph{Hélicoptères multirotors}
		Une des solutions pour éviter d'avoir recours à des systèmes anticouple est de concevoir des machines avec plusieurs rotors tournants dans des sens opposés. Les rotors peuvent être positionnés sur un même axe ou sur des axes séparés.
		
		\begin{center}
		\begin{minipage}[c]{1.0\linewidth}
		\begin{figure}[H]
		\begin{minipage}[c]{0.5\linewidth}
		\centering
		\includegraphics[trim={0 20px 0 0},clip, width=0.95\linewidth]{01-EtudeAeronefs/img/KamowK32A.jpg}
		\legende{Un hélicoptère multirotor coaxiaux : Kamov K32}{img:KamowK32A}
		\end{minipage}
		\begin{minipage}[c]{0.5\linewidth}
		\centering
		\includegraphics[width=0.95\linewidth]{01-EtudeAeronefs/img/CH-147F.jpg}
		\legende{Un hélicoptère multirotor en tandem : Boeing CH-147F}{img:CH-147F}
		\end{minipage}
		\end{figure}
		\end{minipage}
		\end{center}
		
		\paragraph{Convertibles}
		Les convertibles \anglais{tiltrotor} sont en quelque sort des hybrides entre des appareils à voilure tournante et des appareils à voilure fixe.
		
		Grâce à des systèmes qui permettent à leurs rotors de basculer d'une position verticale à une position horizontale, ces appareils cherchent à combiner les avantages de l'hélicoptère (vol stationnaire, décollage et atterrissage verticaux) à ceux des appareils à voilure fixe (vitesse de croisière). Cela se fait, cependant, au prix d'une complexité technique qui rend l'exploitation de ces appareils couteuse et délicate.
		
	\begin{figure}[H]
  	\centering
    \includegraphics[width=0.4\textwidth]{01-EtudeAeronefs/img/tiltrotor.jpg}
  	\legende{Un convertible : le V22 Osprey}{img:tiltrotor}
	\end{figure}	
		
		\paragraph{Autogires}
		L'autogire \anglais{autogyro} est un aérodyne dont la sustentation est assurée grâce à un rotor entrainé par le vent relatif. Les autogires doivent donc dispose d'un mode de propulsion (typiquement une hélice entrainée par un moteur à pistons) qui assure la propulsion de l'appareil sur le plan horizontal.
		
		\histoire{L'autogire à précédé l'hélicoptère. Le premier vol d'un autogire a été effectué en 1923 par l'ingénieur espagnol Juan de la Cierva, contre 1936 pour le premier vol réellement contrôlé pour un hélicoptère.}
		
	\begin{figure}[H]
  	\centering
    \includegraphics[width=0.4\textwidth]{05-Histoire/img/CiervaC6.jpg}
  	\legende{Le C6 de Juan de la Cierva (1924)}{img:CiervaC6}
	\end{figure}	
	
	\paragraph{Girodynes}
	Le girodyne \anglais{gyrodyne} est en quelque sorte un hybride entre un hélicoptère et un autogire. Le rotor principal est en effet entrainé par le moteur, tandis que la traction est obtenue par un ou plusieurs autres moteurs.
	
	\begin{figure}[H]
  	\centering
    \includegraphics[width=0.4\textwidth]{01-EtudeAeronefs/img/Eurocopter-X3.jpg}
  	\legende{Un girodyne : l'Eurocopter X3}{img:Eurocopter-X3}
	\end{figure}
	
	Cette configuration présente des avantages en terme de consommation (le plan de rotation du rotor de sustentation reste parallèle au déplacement, ce qui réduit la trainée) tout en conservant les possibilité de décollage et d'atterrissage vertical d'un hélicoptère.
		
	\subsubsection{Aéronef à voilure souple}
	Les aéronefs à voilure souple \anglais{flexible wings} regroupent des aéronefs dont la voilure est généralement gonflée par l'air : parapente, paramoteur, parachute sportif moderne... Les deltaplanes rentrent également dans cette catégorie.
	
	\info{Le parachute, dans sa version initiale (coupole), se trouve à la limite de la définition d'un aéronef. En effet, ce type de parachute ne produit pas de portance (uniquement de la traînée) et sa capacité de manœuvre est très limitée.}
	
	\begin{center}
	\begin{minipage}[c]{1.0\linewidth}
	\begin{figure}[H]
	\begin{minipage}[c]{0.5\linewidth}
	\centering
	\includegraphics[width=0.7\linewidth]{01-EtudeAeronefs/img/paraMili.jpg}
	\legende{Un parachute coupole}{img:paraMili}
	\end{minipage}
	\begin{minipage}[c]{0.5\linewidth}
	\centering
	\includegraphics[width=0.7\linewidth]{01-EtudeAeronefs/img/paraSport.jpg}
	\legende{Un parachute sportif moderne}{img:paraSport}
	\end{minipage}
	\end{figure}
	\end{minipage}
	\end{center}
	
	\histoire{André-Jacques \textbf{Garnerin}, un ingénieur français, effectue le premier saut réussi en parachute en 1797, en sautant depuis un ballon au dessus de Paris.}
	
\subsection{Engins spatiaux}
	Cette catégorie regroupe tous les appareils destinés à évoluer pour la totalité ou seulement une partie de leur mission en dehors de l'atmosphère terrestre.
	
	On distingue principalement les lanceurs, capables d'évoluer dans l'atmosphère terrestre et dans l'espace, et les engins purement spatiaux, qui n'évoluent que dans l'espace.

	\subsubsection{Lanceurs}
	Les lanceurs spatiaux sont capables d'évoluer dans l'atmosphère terrestre pour le décollage mais également pour le retour sur Terre lorsque la mission le nécessite. Ils sont également en mesure d'évoluer en dehors de l'atmosphère pour assurer la phase spatiale de leur mission : mise en orbite de satellite, rendez-vous spatial pour rejoindre un satellite déjà en orbite...
	
	La mission des lanceurs est de fournir la poussée pour atteindre les vitesses nécessaires à la mise en orbite de satellites ou à l'envoi de sondes spatiales. Pour cela, le lanceur doit disposer de l'ensemble des systèmes permettant de le diriger mais également emporter l'ensemble des combustibles et comburants (appelés \textit{ergols}) nécessaires au fonctionnement de sa propulsion dans et en dehors de l'atmosphère.
	
		\paragraph{Fusées}
		Les fusées \anglais{rocket} sont les premiers types de lanceurs utilisés. Dérivés des missiles, ils utilisent des moteurs fusées pour accomplir leurs missions.
		
		Les fusées ont permis l'envoi des premiers satellites, des premiers humains dans l'espace et la réalisation de missions lunaires.
		
	\begin{figure}[H]
  	\centering
    \includegraphics[width=0.4\textwidth]{01-EtudeAeronefs/img/ariane5.jpg}
  	\legende{La fusée européenne Ariane 5 sur son pas de tir}{img:ariane5}
	\end{figure}
		
		Les évolutions récentes des fusées pousse vers des ensembles réutilisables afin de contenir les coûts des lancements.	
		
		\paragraph{Navettes spatiales}
		Le terme navette spatiale \anglais{space shuttle} désigne un vaisseau spatial conçu pour revenir sur Terre en planant, à la façon d'un avion et destiné à être réutilisé.
		
		Ce type de vaisseau offre généralement un volume d'emport très supérieur à celui des fusées. Initialement, l'objectif était également de réduire le cout des lancements en permettant la réutilisation de la majeure partie du vaisseau.
		
		\begin{center}
		\begin{minipage}[c]{1.0\linewidth}
		\begin{figure}[H]
		\begin{minipage}[c]{0.3\linewidth}
		\centering
		\includegraphics[width=0.95\linewidth]{01-EtudeAeronefs/img/NavetteSpatialeNasa.jpg}
		\legende{La navette américaine}{img:NavetteSpatialeNasa}
		\end{minipage}
		\begin{minipage}[c]{0.7\linewidth}
		\centering
		\includegraphics[width=0.95\linewidth]{01-EtudeAeronefs/img/Bourane.jpg}
		\legende{Bourane, la navette soviétique présentée ici sur son avion porteur, l'Antonov 225}{img:Bourane}
		\end{minipage}
		\end{figure}
		\end{minipage}
		\end{center}
		
		\histoire{Le projet de navette spatiale le plus connu est celui de la navette américaine. L'union soviétique avait également développé sa propre navette à la fin des années 1980. Ce programme, dénommé \textit{Bourane}, a été abandonné après le premier vol orbital de la navette, réalisé sans équipage.}
		
	\subsubsection{Engins spatiaux}
		\paragraph{Satellites}
		Un satellite artificiel est un objet créé par l'Homme et envoyé dans l'espace grâce à un lanceur. Le satellite orbite (tourne) autour d'une planète (la Terre ou Mars, par exemple) ou de l'un de ses satellite naturel (la Lune par exemple).
		
		Les satellites couvrent des missions très variées. Ils peuvent par exemple servir pour la télécommunication (téléphonie ou télévision par satellite), l'observation de la Terre, à des fins scientifiques ou militaire (satellite espion), ou encore servir de télescopes. 
		
		\begin{center}
		\begin{minipage}[c]{1.0\linewidth}
		\begin{figure}[H]
		\begin{minipage}[c]{0.5\linewidth}
		\centering
		\includegraphics[width=0.95\linewidth]{01-EtudeAeronefs/img/hubble.jpg}
		\legende{Le téléscope Hubble}{img:hubble}
		\end{minipage}
		\begin{minipage}[c]{0.5\linewidth}
		\centering
		\includegraphics[width=0.95\linewidth]{01-EtudeAeronefs/img/iridium.jpg}
		\legende{Un satellite de télécommunication Iridium}{img:iridium}
		\end{minipage}
		\end{figure}
		\end{minipage}
		\end{center}
		
		\histoire{Le premier satellite artificiel, \textbf{Spoutnik 1}, a été lancé par l'URSS en 1957. \\ La France lancera son premier satellite, \textbf{Asterix}, en 1965}
		
		\info{On estime qu'en 2024, il y a environ 10\,000 satellites en orbite terrestre.}
		
		\subparagraph{Les stations spatiales}
		La plupart des satellites ne sont pas habités, mais il en existe qui peuvent accueillir des équipages. C'est le cas de la station spatiale internationale (ISS). Ces satellites permettent la réalisation d'expériences scientifiques.
		
	\begin{figure}[H]
  	\centering
    \includegraphics[width=0.4\textwidth]{01-EtudeAeronefs/img/iss.jpg}
  	\legende{La station spatiale internationale en 2021}{img:iss}
	\end{figure}
		
		Ce type de satellite nécessite de nombreux équipements pour permettre la vie à bord (système de conditionnement pour assurer la pressurisation et un atmosphère respirable...) mais également d'y venir et de les quitter : système d'amarrage permettant à un vaisseau de se connecter à la station, sas pour sortie spatiale...
		
		\histoire{La première station spatiale, Saliout, a été mise en orbite par l'URSS en 1971. \\ Depuis 1998, la Station Spatiale Internationale est la principale station en orbite. Elle est issue de la collaboration d'agences spatiales du monde entier.}
		
		\paragraph{Sondes}
		Les sondes spatiales \anglais{uncrewed spacecraft} sont des véhicules spatiaux qui quittent l'orbite terrestre pour étudier d'autres objets célestes : autres planètes, soleil, ou encore comète.
		
		Les sondes spatiales peuvent prendre des formes très différentes selon que leur mission sera constitué de simples survols du ou des corps célestes étudiés ou si au contraire elles doivent s'y poser.
		
		\begin{center}
		\begin{minipage}[c]{1.0\linewidth}
		\begin{figure}[H]
		\begin{minipage}[c]{0.5\linewidth}
		\centering
		\includegraphics[width=0.95\linewidth]{01-EtudeAeronefs/img/Voyager2.jpg}
		\legende{La sonde Voyager 2}{img:Voyager2}
		\end{minipage}
		\begin{minipage}[c]{0.5\linewidth}
		\centering
		\includegraphics[width=0.95\linewidth]{01-EtudeAeronefs/img/curiosity.jpg}
		\legende{L'atterisseur martien Curiosity}{img:curiosity}
		\end{minipage}
		\end{figure}
		\end{minipage}
		\end{center}
		
		\histoire{La sonde spatiale la plus éloignée de la Terre est le sonde Voyager 1. Cette sonde, lancée en 1977 est aujourd'hui située à plus de 25 milliards de kilomètres de la Terre, et continue, malgré son âge et la distance, à envoyer des données scientifiques vers la Terre.}

\subsection{Des grandes familles d'aéronefs}		
\subsubsection{Les ULM}

Les \acrshort{ulm} (\acrlong{ulm}) \anglais{ultralight aircraft} sont un ensemble d'aéronefs motorisés de faible masse. De par cette caractéristique, ils bénéficient de facilités quant à leurs conditions de conception et d'entretien. L'obtention d'une licence de pilote ULM est également simplifiée par rapport aux licences pour des appareils plus lourds (une licence ULM peut s'obtenir après environ 20h de vol). \\

Il existe une grande diversité d'ULM. Il est ainsi possible de piloter des ULM aérostats ou aérondynes, à voilures fixes, tournantes ou souples. En France, ils sont classés en 6 catégories. La licence de pilote d'ULM est commune à toutes les classes, cependant, il faut obtenir une qualification propre à chaque classe, qui permet au pilote de se familiariser avec un instructeur aux particularités propre à chaque type d'aéronef. \\

Chaque classe dispose de ses propres limitations en termes de masse et puissance maximale. Les ULM sont monoplace ou biplace.

	\paragraph{Classe 1 : paramoteur}

	Cette classe est composée d'aérodynes à voilure souple. Elle prend la forme d'un voile type voile de parapente. Le système propulsif, composé d'un moteur et d'une hélice carénée, est soit porté directement dans le dos par le pilote, soit fixé sur un chariot.  
	
	\begin{figure}[H]
  	\centering
    \includegraphics[width=0.4\textwidth]{01-EtudeAeronefs/img/ULM_Classe_1.jpg}
  	\legende{ULM classe 1}{img:ulmClasse1}
	\end{figure}	
	
	\paragraph{Classe 2 : pendulaire}
	Cette classe est composée d'aérodynes constitués de chariots sur lequel est fixé une aile delta. Le nom pendulaire vient du fait que les changements de trajectoires sont obtenus par déplacement du centre de gravité.	
	
	\begin{figure}[H]
  	\centering
    \includegraphics[width=0.4\textwidth]{01-EtudeAeronefs/img/ULM_Classe_2.jpg}
  	\legende{ULM classe 2}{img:ulmClasse2}
	\end{figure}	

	\paragraph{Classe 3 : multiaxes}
	Cette classe est composée d'aérodynes à voilure fixe, qui s'apparentent à des avions traditionnels. Ils est par ailleurs parfois difficile de distinguer un ULM  multiaxe moderne d'un avion biplace.
	
	\info{Aujourd'hui la frontière entre avion et ULM mutliaxes est très réduite. A un telle point que certains fabricants d'avion proposent un même modèle sous la réglementation avion ou ULM en fonction de la masse maximale inscrite sur les documents. C'est le cas de l'ULM présenté en photo ci dessous (commercialisé sous le nom Bristell Classic sous réglementation ULM et Bristel B23 en réglementation avion).}
	
	\begin{figure}[H]
  	\centering
    \includegraphics[width=0.4\textwidth]{01-EtudeAeronefs/img/ULM_Classe_3.jpg}
  	\legende{ULM classe 3}{img:ulmClasse3}
	\end{figure}	

	\paragraph{Classe 4 : autogyre}
	Cette classe est composée d'aérodynes à voilure tournante de type autogyres.

	\begin{figure}[H]
  	\centering
    \includegraphics[width=0.4\textwidth]{01-EtudeAeronefs/img/ULM_Classe_4.jpg}
  	\legende{ULM classe 4}{img:ulmClasse4}
	\end{figure}	
	
	\paragraph{Classe 5 : ballon dirigeable}
	Il s'agit d'aérostats dirigeables dont le volume de l'enveloppe est inférieur à 900 m² pour un  ballon à l'hélium et 2000 m² pour les ballons à air chaud.	
	
	\begin{figure}[H]
  	\centering
    \includegraphics[width=0.4\textwidth]{01-EtudeAeronefs/img/ULM_Classe_5.jpg}
  	\legende{ULM classe 5}{img:ulmClasse5}
	\end{figure}	
		
	\paragraph{Classe 6 : hélicoptère}	
	Dernière née des catégories d'ULM (créée en 2012), cette classe est composée d'aérodynes à voilure tournante de type hélicoptères.
	
	\begin{figure}[H]
  	\centering
    \includegraphics[width=0.4\textwidth]{01-EtudeAeronefs/img/ULM_Classe_6.jpg}
  	\legende{ULM classe 6}{img:ulmClasse6}
	\end{figure}


\subsubsection{Les drones}
Les drones sont des aéronefs pilotés à distance. 

Les drones peuvent être tout type d'aéronef : voilure tournante (typiquement multirotors), voilure fixe, aérostat... Les drones peuvent être libres ou captifs (retenus au sol par un câble).

Il existe également une grande diversité dans les dimensions du drones. Cela va d'appareils mesurant quelques dizaines de centimètre de côté et quelques centaines de grammes à des machines de plusieurs tonnes et de grande envergure.

		\begin{center}
		\begin{minipage}[c]{1.0\linewidth}
		\begin{figure}[H]
		\begin{minipage}[c]{0.5\linewidth}
		\centering
		\includegraphics[width=0.95\linewidth]{01-EtudeAeronefs/img/DJI-Air3.jpg}
		\legende{Un drone multirotors, le DJI Air 3 (720~g, environ 40~cm d'envergure)}{img:DJI-Air3}
		\end{minipage}
		\begin{minipage}[c]{0.5\linewidth}
		\centering
		\includegraphics[width=0.95\linewidth]{01-EtudeAeronefs/img/nEUROn.jpg}
		\legende{Le nEUROn de Dassault Aviation (7~tonnes, 12.5~m d'envergure)}{img:nEUROn}
		\end{minipage}
		\end{figure}
		\end{minipage}
		\end{center}

Un système drone est toujours composé de systèmes au sol permettant le contrôle du drone ainsi que du drone lui même. Le drone doit comporter de nombreux systèmes pour assurer son autonomie énergétique, le contrôle de son attitude, le contrôle de sa trajectoire et le suivi de son plan de vol ou encore la liaison dans les 2 sens avec le système de pilotage à distance. 

Enfin, il faut ajouter à ces systèmes les composants propres à la mission du drone : capteurs (photo, infrarouge, radio...), systèmes de largage ou d'épandage...
