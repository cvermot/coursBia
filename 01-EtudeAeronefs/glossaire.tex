\newglossaryentry{aéronef}
{
    name=aéronef,
    description={Appareil capable de s'élever et de déplacer dans l'atmosphère}
}
\newglossaryentry{aérostat}
{
    name=aérostat,
    description={Aéronef plus léger que l'air}
}
\newglossaryentry{aérodyne}
{
    name=aérodyne,
    description={Aéronef dont la sustentation est assurée par la portance d'une voilure fixe ou tournante}
}
\newglossaryentry{badin}
{
    name=badin,
    description={Instrument servant à mesurer la vitesse d'un avion par rapport à l'air. Synonyme de badin}
}
\newglossaryentry{anémomètre}
{
    name=anémomètre,
    description={Instrument servant à mesurer la vitesse d'un avion par rapport à l'air}
}

\newglossaryentry{pitot}
{
    name=tube pitot,
    description={Capteur permettant de fournir l'information de vitesse à l'anémomètre}
}
\newglossaryentry{altimètre}
{
    name=altimètre,
    description={Instrument servant à mesurer l'altitude à partir de la pression atmosphérique}
}
\newglossaryentry{anéroïde}
{
    name=capsule anéroïde,
    description={Element capteur d'un altimètre}
}


\newacronym{ulm}{ULM}{Ultra Léger Motorisé}
\newacronym{vno}{VNO}{Velocity Normal Operating}
\newacronym{vne}{VNE}{Velocity Never Exceed}
\newacronym{vfe}{VFE}{Velocity Flaps Extended}
\newacronym{vle}{VLE}{Velocity Landing gear Extended}