\newglossaryentry{aéronef}
{
    name=aéronef,
    description={Appareil capable de s'élever et de déplacer dans l'atmosphère}
}
\newglossaryentry{aérostat}
{
    name=aérostat,
    description={Aéronef plus léger que l'air}
}
\newglossaryentry{aérodyne}
{
    name=aérodyne,
    description={Aéronef dont la sustentation est assurée par la portance d'une voilure fixe ou tournante}
}
\newglossaryentry{badin}
{
    name=badin,
    description={Instrument servant à mesurer la vitesse d'un aéronef par rapport à l'air. Synonyme d'anémomètre}
}
\newglossaryentry{anémomètre}
{
    name=anémomètre,
    description={Instrument servant à mesurer la vitesse d'un aéronef par rapport à l'air}
}
\newglossaryentry{variomètre}
{
    name=variomètre,
    description={Instrument servant à mesurer la vitesse verticale (montée ou descente) d'un aéronef}
}

\newglossaryentry{pitot}
{
    name=tube Pitot,
    description={Capteur permettant de fournir l'information de vitesse à l'anémomètre}
}
\newglossaryentry{altimètre}
{
    name=altimètre,
    description={Instrument servant à mesurer l'altitude à partir de la pression atmosphérique}
}
\newglossaryentry{horizon artificiel}
{
    name=horizon artificiel,
    description={Instrument servant à afficher l'assiette et l'inclinaison d'un aéronef}
}
\newglossaryentry{indicateur de virage}
{
    name=indicateur de virage,
    description={Instrument servant à afficher l'inclinaison et le dérapage d'un aéronef}
}
\newglossaryentry{conservateur de cap}
{
    name=conservateur de cap,
    description={Instrument servant à afficher le cap et offrant un confort de lecture supérieur à la boussole}
}
\newglossaryentry{compas magnétique}
{
    name=compas magnétique,
    description={Instrument affichant un cap et utilisant le champ magnétique terreste comme référence}
}
\newglossaryentry{boussole}
{
    name=boussole,
    description={Voir compas magnétique}
}
\newglossaryentry{anéroïde}
{
    name=capsule anéroïde,
    description={Element capteur d'un altimètre}
}
\newglossaryentry{glasscockpit}
{
    name=glass cockpit,
    description={Planche de bord tout écran}
}

\newacronym{rac}{RAC}{rotor anticouple}

\newacronym{ulm}{ULM}{Ultra Léger Motorisé}

\newacronym{vno}{VNO}{Velocity Normal Operating}
\newacronym{vne}{VNE}{Velocity Never Exceed}
\newacronym{vfe}{VFE}{Velocity Flaps Extended}
\newacronym{vle}{VLE}{Velocity Landing gear Extended}

\newacronym{qfe}{QFE}{Pression au niveau de l'aérodrome}
\newacronym{qnh}{QNH}{Pression au niveau de la mer}

\newacronym{efis}{EFIS}{Electronic flight instrument system}