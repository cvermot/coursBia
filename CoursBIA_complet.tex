\documentclass[a4paper,12pt,oneside]{report} % définition des différents parametres. Le dernier peut-être de type report ou article (modifie le chapitrage)
	\usepackage[french]{babel} %déclaration de la langue du document
	\usepackage[utf8]{inputenc} %déclaration du jeu de caractéres à utiliser
	\usepackage[top=2.5cm, bottom=2.5cm, left=2.5cm, right=2.5cm]{geometry} %permet de modifier les marges
	\usepackage{graphicx} %package nécessaire pour l'insertion d'images
	\usepackage{graphics} %package nécessaire pour l'insertion d'images 
	\usepackage{wrapfig} %package nécessaire pour l'insertion d'images
	\usepackage{amsmath} %package mathématiques (fonction \dfrac notamment)
	\usepackage{multicol} %package pour écire sur plusieurs colonnes
	\usepackage{hyperref} %package pour ajouter des liens hypertextes dans le ficher de sortie
	\usepackage{array} % meilleur gestion des tableaux
	%\usepackage{textcomp}%pour avoir le symbole euro --> \texteuro
	\usepackage[toc]{glossaries}
	\usepackage{animate}
	\usepackage{tikz}
	\usepackage{rotating}
	\usepackage{float}
	%\usepackage{pdfpages}
	%\usepackage{siunitx}
	\usepackage{float}
	\usepackage{xcolor}
	\usepackage{color}
	%\usepackage{colortbl}
	\usepackage[notlot]{tocbibind}
	\usepackage{etoc}
	\usepackage{longtable}
	\usepackage{textpos}
	\usepackage{fontawesome5}
	
	\usepackage{phdthesis}
	%\setlength{\parskip}{0.8ex}
	
	%Utilisation de la police OpenDyslexic => compiler avec LuaLaTeX
	%\usepackage{fontspec}
 	%\setmainfont{OpenDyslexic}
	
	\newcolumntype{L}[1]{>{\raggedright}m{#1}}
		
	\title{Cours BIA} %titre du document
	\author{Clément \textbf{Vermot-Desroches}} %auteurs du document
	
	\setlongtables
	
	\makeindex
	\makeglossaries
	%\glossarystyle{long3col}	
	% pour personnaliser la façon dont sont imprimés les entrées du glossaire
	\renewcommand{\glstextformat}[1]{\textsf{#1}}
	%\renewcommand{\glstextformat}[1]{#1*}
	%\renewcommand{\glstextformat}[1]{\color{blue!70!black}\bfseries#1}
	
	\newglossaryentry{aéronef}
{
    name=aéronef,
    description={Appareil capable de s'élever et de déplacer dans l'atmosphère}
}
\newglossaryentry{aérostat}
{
    name=aérostat,
    description={Aéronef plus léger que l'air}
}
\newglossaryentry{aérodyne}
{
    name=aérodyne,
    description={Aéronef dont la sustentation est assurée par la portance d'une voilure fixe ou tournante}
}
%\newacronym{gcd}{GCD}{Greatest Common Divisor}
	
	%Numérotation des paragraphs
	\renewcommand{\theparagraph}{\arabic{chapter}.\arabic{section}.\arabic{subsection}.\arabic{subsubsection}.\arabic{paragraph}}
	\renewcommand{\thesubparagraph}{\arabic{chapter}.\arabic{section}.\arabic{subsection}.\arabic{subsubsection}.\arabic{paragraph}.\arabic{subparagraph}}
	\setcounter{secnumdepth}{5}
	
	%%%% debut macro %%%%
	\newenvironment{changemargin}[2]{\begin{list}{}{%
	\setlength{\topsep}{0pt}%
	\setlength{\leftmargin}{0pt}%
	\setlength{\rightmargin}{0pt}%
	\setlength{\listparindent}{\parindent}%
	\setlength{\itemindent}{\parindent}%
	\setlength{\parsep}{0pt plus 1pt}%
	\addtolength{\leftmargin}{#1}%
	\addtolength{\rightmargin}{#2}%
	}\item }{\end{list}}
	%%%% fin macro %%%%

	
	%page de garde
%	\makeatletter
%		\def\clap#1{\hbox to 0pt{\hss #1\hss}}%
%		\def\ligne#1{%
%		\hbox to \hsize{%
%		\vbox{\centering #1}}}%
%		\def\haut#1#2#3{%
%		\hbox to \hsize{%
%		\rlap{\vtop{\raggedright #1}}%
%		\hss
%		\clap{\vtop{\centering #2}}%
%		\hss
%		\llap{\vtop{\raggedleft #3}}}}%
%		\def\bas#1#2#3{%
%		\hbox to \hsize{%
%		\rlap{\vbox{\raggedright #1}}%
%		\hss
%		\clap{\vbox{\centering #2}}%
%		\hss
%		\llap{\vbox{\raggedleft #3}}}}%
%		\def\maketitle{%
%		\thispagestyle{empty}\vbox to \vsize{%
%		\haut{}{\@blurb}{}
%		\vfill
%		\vspace{1cm}
%		\begin{flushleft}
%		\usefont{OT1}{ptm}{m}{n}
%		\huge \@title
%		\end{flushleft}
%		\par
%		\hrule height 4pt
%		\par
%		\begin{flushright}
%		\usefont{OT1}{phv}{m}{n}
%		\Large \@author
%		\par
%		\end{flushright}
%		\vspace{1cm}
%		\vfill
%		\vfill
%		\bas{}{\@date}{}
%		}%
%		\cleardoublepage
%		}
%		\def\date#1{\def\@date{#1}}
%		\def\author#1{\def\@author{#1}}
%		\def\title#1{\def\@title{#1}}
%		\def\location#1{\def\@location{#1}}
%		\def\blurb#1{\def\@blurb{#1}}
%		\date{Avril 2012 - version 1.0}
%		\author{}
%		\title{}
%		
%		\makeatother
%		\title{Cours radioamateur - classe 2 \& 3}
%		\author{F4HAJ}
%		\includegraphics[scale=1]{./img/IC-7800.jpg}
%		\blurb{%
%			Test
%		}%	
	%fin page de garde
		
	
	\begin{document}
	
	\begin{titlepage}
		\begin{center} 
			~
			\vfill
			\begin{flushleft}
				\huge \textbf{Cours BIA}
			\end{flushleft}
			\par
			\hrule height 4pt
			\par
			\begin{flushright}
				\Large \textbf{Clément \textsc{Vermot-Desroches}}
				\par
			\end{flushright}
			\vspace{2cm}
			%\includegraphics[width=\textwidth]{./img/img_couverture.png} 
			\vfill
			\vfill
		\end{center}
			\begin{textblock*}{30mm}[0,0](-20mm,-25mm)
				%\includegraphics[width=3cm]{./img/International_symbol.png} 
			\end{textblock*}
		\begin{center}
			http://f4haj.net
			
			Février 2025 - version 0.1
		\end{center} 
	\end{titlepage} 	
	

	\tableofcontents
	\newpage	
	
	\section{Préface}
	\alert{test\\test\\test}
	\info{test}
	\question{test}
	\histoire{test}

	Todo

	%Ce document était à la base un simple mémento destiné à mon usage propre en vue du passage des licences classe 2 et 3 d'opérateur du service amateur. Il est devenu, au fur et à mesure de mon apprentissage, un document résumant de façon synthétique (je pense), les principales notions nécessaires à la préparation de ces licences. Ce document m'a permis de préparer et de passer simultanément et avec succès les examens de législation et de technique, et ceci en seulement 2 mois de préparation.

	%J'ai décidé de le diffuser car je trouve qu'il n'est pas facile de trouver sur Internet des cours à la fois complets et synthétiques pour la préparation de la licence radioamateur française.

	%Je ne prétend pas que l'approche adoptée dans ce document conviendra à tout le monde. Ce document regroupe les principales notions et formules à connaitre. Il peut servir de base pour la préparation d'une licence ou encore de mémento pour le maintient des connaissances après l'obtention de la licence. Je continue de penser que l'adhésion à un radioclub demeure indispensable, ne serais-ce que pour donner un certaines synergie à cet apprentissage passionnant mais qui peut parfois paraître fastidieux. Par ailleurs, les informations et conseils que pourront vous fournir les membres d'un radioclub seront toujours utiles et bienvenus, y compris après l'obtention de la licence (conseil sur le matériel, le montage d'une antenne...).

	%Enfin, ce document est destiné à évoluer. Je souhaite ajouter de nouvelles notions prochainement pour ceux qui souhaiteraient aller plus loin. Par ailleurs, je suis ouvert à toutes critiques ou commentaires constructifs concernant ce document, sur le fond comme sur la forme. N'hésitez donc pas à me contacter concernant toute erreur, omission, faute d'orthographe ou manque dans ce documents, à l'adresse contact@f4haj.net.
	
	\section{Organisation du cours sur l'année}
	
	\usetikzlibrary{timeline}
	\noindent\begin{tikzpicture}[thick,scale=0.69, every node/.style={scale=0.69},timespan={}]
     \fill[fill=gray!50, opacity=0.20, rounded corners=0.5cm] (12.4,-2.5) rectangle ++(7.6,6);
     \draw[double,fill,line width=0.15mm] (16.2,-2.5) circle [radius=0.08];
     \draw[line width=0.15mm] (16.2,-2.5) -- (16.2,-3.15) ;%node {Vol};
     \draw (16.2,-3.5) node {\faPlaneDeparture~Vols};
	
	 \timeline[custom interval=true]{Sep, Oct, Nov, Déc, Jan, Fév, Mar, Avr, Mai, Jui}
	 
	 \begin{phases}
	 %\initialphase{involvement degree=2.75cm,phase color=black}
	 \phase{between week=1 and 1 in 0.1,involvement degree=2cm,phase color=black}
	 \phase{between week=2 and 5 in 0.1,involvement degree=4.5cm,phase color=red!80}
	 \phase{between week=4 and 6 in 0,involvement degree=4cm,phase color=blue!80}
	 \phase{between week=6 and 8 in -0.2,involvement degree=4cm,phase color=green!80}
	 \phase{between week=7 and 9 in 0.1,involvement degree=4cm,phase color=yellow!80}
	 \phase{between week=8 and 9 in 0.5,involvement degree=2.25cm,phase color=orange!80}
	 \phase{between week=9 and 10 in 0.6,involvement degree=2cm,phase color=black}
	 
	 \phase{between week=3 and 3 in 0.6,involvement degree=2cm,phase color=violet}
	 \phase{between week=6 and 7 in 0.7,involvement degree=2cm,phase color=violet}
      \phase{between week=10 and 10 in 1.0,involvement degree=2cm,phase color=violet}
	%\phase{between week=1 and 2 in -0.5,involvement degree=2.25cm}	
	\end{phases} 
	
	 \addmilestone{at=phase-1.90,direction=90:2.0cm,text={\faSchool~Rentrée},text options={above}}
	 \addmilestone{at=phase-2.90,direction=90:2.0cm,text={\hyperlink{aeronef}{\faSpaceShuttle~ Étude des aéronefs et des engins spatiaux (11h)}},text options={above}}
	 \addmilestone{at=phase-3.270,direction=270:1.5cm,text={\hyperlink{nav}{\faMap[regular]~Navigation, réglementation, sécurité des vols (9h)}},text options={below}}
	 \addmilestone{at=phase-4.90,direction=90:1.5cm,text={\hyperlink{meteo}{\faCloudSun~Météorologie et aérologie (9h)}},text options={above}}
	 \addmilestone{at=phase-5.270,direction=270:2.5cm,text={\hyperlink{aerodynamique}{\faPaperPlane[regular]~Aérodynamique, aérostatique et principes du vol (9h)}},text options={below}}
	 \addmilestone{at=phase-6.90,direction=90:3.1cm,text={\hyperlink{histoire}{\faBook~Histoire et culture de l’aéronautique et du spatial (complement) (3h)}},text options={above}}
	 \addmilestone{at=phase-7.270,direction=270:1.0cm,text={\faGraduationCap~BIA},text options={below}}
	 
	 \addmilestone{at=phase-8.270,direction=270:1.5cm,text={\faPlane~Visite aéroclub},text options={below}}
	 \addmilestone{at=phase-9.90,direction=90:1.9cm,text={\faFighterJet~Vol simulé},text options={above}}
      \addmilestone{at=phase-10.90,direction=90:1.9cm,text={\faRocket~(Micro-fusées)},text options={above}}
	 
	\end{tikzpicture}

	\section{Licence}
	Ce document est diffusé sous licence Creative Commons 4.0 by-nc-sa dont les termes sont disponibles sur le site CreativeCommons.org\footnote{Termes disponibles à cette adresse : \url{http://creativecommons.org/licenses/by-nc-sa/4.0/}}. De façon synthétique, cette licence \textbf{vous autorise à redistribuer et à modifier ce document tant que vous le souhaitez tant que vous me citez en tant qu'auteur original} (un lien vers la version vous ayant servit de base est souhaité). En cas de diffusion de la version modifiée, vous avez l'obligation de la diffuser sous la même licence. Toute utilisation commerciale est en revanche proscrite. Si vous souhaitez \textbf{faire un usage dépassant le cadre de cette licence} (donc utilisation commerciale, diffusion sous une autre licence...),vous devez \textbf{impérativement obtenir mon autorisation expresse écrite}. Pour cela vous pouvez me contacter à l'adresse contact@f4haj. Je tiens par ailleurs à votre disposition le fichier source \LaTeX{} de ce document sur simple demande.
	

	\chapter{Étude des aéronefs et des engins spatiaux}

\usetikzlibrary {backgrounds,mindmap}
\begin{tikzpicture}
  [root concept/.append style={concept color=blue!80,minimum size=2cm},
   level 1 concept/.append style={sibling angle=180},
   level 2 concept/.append style={sibling angle=120},
   level 3 concept/.append style={sibling angle=90},
   mindmap]
  \node [concept] (aeronef) {Aéronef}
    [clockwise from=0]
    child[concept color=red] { node[concept] (aerodyne) {Aérodyne}
    		[clockwise from=90]
      child[concept color=green] { node[concept] {Voilure fixe} 
      	[clockwise from=-180]
      	child[concept color=yellow] { node[concept] {Avion} }
      	child[concept color=yellow] { node[concept] {Planeur} }
      	child[concept color=yellow] { node[concept] {Deltaplane} }
      }
      child[concept color=green] { node[concept] {Voilure tournante} 
      	[clockwise from=45]
      	child[concept color=yellow] { node[concept] {Hélico} }
      	child[concept color=yellow] { node[concept] {Autogyre} }
      }
      child[concept color=green] { node[concept] {Deltaplane} }	
      }
    child[concept color=red] { node[concept] (aerostat) {Aérostat}}
    ;
\end{tikzpicture}	

		\section{Classification des aéronefs}
Un \gls{aéronef} est un appareil capable de s'élever et de se mouvoir au sein de l'atmosphère terrestre. On divise les aéronefs en 2 grandes familles :
\begin{itemize}
	\item les  \gls{aérostat}s, qui sont des appareils plus légers que l'air,
	\item les  \gls{aérodyne}s, qui sont plus lourds que l'air.
\end{itemize}

Dans cette partie, nous étudierons les aéronefs mais également les engins spatiaux. Ceux ci ne peuvent être qualifiés d'aéronefs, car, bien que certains d'entre eux puissent se déplacer dans l'atmosphère, ils peuvent également se mouvoir en dehors de celle-ci.

\subsubsection{Pourquoi classer les aéronefs ?}
Chaque type d'aéronef dispose de propriétés et de contraintes qui lui sont propres. La classification des aéronefs en grand groupes présentant des caractéristiques communes permet de leur associer aisément des notions réglementaires (licence de pilote nécessaire, minima météo, zone de vol autorisées...), techniques (fréquences et mode d'entretien, contraintes de conception) ou administratives (immatriculation, assurance...).

\subsection{Aérostats}
	\subsubsection{Montgolfière}
	La montgolfière est un aérostat gonflé à l'air chaud.
	
	Elle est composé d'un ballon (appelé enveloppe) sous lequel est accroché une nacelle dans laquelle prennent place les passagers et le pilote nommé aérostier. Un bruleur généralement alimenté au gaz permet de chauffer l'air contenu dans le ballon. Le pilote chauffe l'air à l'aide du bruleur pour faire monter la montgolfière. \\
	
	La montgolfière ne dispose d'aucun moyen pour se diriger, elle est entièrement soumise aux vents pour ses déplacements. Cependant, le pilote peut exploiter les variation de sens du vent aux différentes altitudes pour orienter son vol dans une certaine mesure. \\
	
	\histoire{La montgolfière a été le premier aéronef conçu par l'humain. Les frères Montgolfier ont conçu le premier ballon à air chaud et réalisé le premier vol en 1783. La même année, ils font voler des animaux puis Jean-François \textbf{Pilâtre de Rozier} et le Marquis d'Arlandes réalisent le premier vol libre humain.}
	\subsubsection{Ballon à gaz}
	Le ballon à gaz est gonflé avec un gaz plus léger que l'air (hydrogène ou hélium).	
	
	\subsubsection{Dirigeable}
	Le dirigeable ou ballon dirigeable est un ballon à gaz équipés d'une systèmes propulsifs lui permettant de se diriger (aussi bien sur le plan horizontal que vertical). \\
	
	Les premiers dirigeables étaient gonflés à l'hydrogène. Ce gaz est dangereux car très inflammable. Les dirigeables modernes sont désormais gonflés à l'hélium. L'hélium est un gaz sur car ininflammable, mais il est plus cher et plus lourd que l'hydrogène (un ballon à l'hélium nécessitera une enveloppe plus grande qu'un ballon à hydrogène de même capacité).

\subsection{Aérodynes}
	\subsubsection{Avions}
	\subsubsection{Planeurs}
	\subsubsection{Hélicoptères}
	\subsubsection{Parapentes, deltaplanes}

\subsection{Engins spatiaux}
	\subsubsection{Lanceurs}
		\paragraph{Fusées}
		\paragraph{Navettes spatiales}
		
	\subsubsection{Engins spatiaux}
		\paragraph{Satellites}
		\paragraph{Sondes}
		
\subsection{Les ULM}

\subsection{Les drones}
		\label{aeronef}
		\usetikzlibrary{calc,fit}
\tikzset{>=latex} % for LaTeX arrow head
\colorlet{knob}{blue!20!black!40}
\colorlet{mylightblue}{blue!10}
\colorlet{mydarkblue}{blue!30!black}
\tikzstyle{arrow}=[<-,very thick,mydarkblue]
\tikzstyle{vector}=[->,line width=2,green!50!black]

% ANGLE
\newcommand{\getangle}[3]{%
    \pgfmathanglebetweenpoints{\pgfpointanchor{#2}{center}}
                              {\pgfpointanchor{#3}{center}}
    \global\let#1\pgfmathresult  
}

\newcommand{\echelleTikz}{1.0}

% ENGINE
\def\gas{blue!10}
\def\engine#1{
  \def\R{2}
  \def\l{1}
  \def\L{4.6}
  \def\p{1.8}
  \def\P{3.2}
  \def\Ra{.35} % crankshaft
  \def\Rb{.6}  % crankshaft
  \def\Rc{.2}  % crankshaft
  \def\a{40} % wall
  \def\b{30} % rod
  \coordinate (O)   at (0,0);
  \coordinate (CS)  at (#1:\l);
  \coordinate (P)   at (0,{\l*sin(#1)+sqrt(\P^2-(\l*cos(#1))^2)});
  \coordinate (RL)  at (180-\a:\R);
  \coordinate (RR)  at (\a:\R);
  \coordinate (TL)  at ($(RL)+(0,\L)$);
  \coordinate (TR)  at ($(RR)+(0,\L)$);
  \coordinate (T)   at ($(TL)!.5!(TR)$);
  \coordinate (PL)  at ($(RL)+(0,{\l*(1.4+sin(#1))})$);
  
  \coordinate (PR)  at ($(PL-|RR)+(0,\p)$);
  \coordinate (S)   at ($(T)+(0,.8)$);
  \coordinate (VLo) at ($(TL)!.2!(S)$);
  \coordinate (VL)  at ($(TL)!.4!(S)$);
  \coordinate (VLm) at ($(TL)!.6!(S)$);
  \coordinate (VRo) at ($(TR)!.2!(S)$);
  \coordinate (VR)  at ($(TR)!.4!(S)$);
  \coordinate (VRm) at ($(TR)!.6!(S)$);
  \getangle{\c}{CS}{P}
  \getangle{\vl}{VLo}{VLm}
  \getangle{\vr}{VRo}{VRm}
  
  % GAS
  \fill[\gas,draw=white,line width=3] (PL) -| (TR) -- (S) -- (TL) -- cycle;
  
  % CRANKSHAFT
  \draw[thick,mydarkblue,top color=blue!30!black!40,bottom color=blue!30!black!10,shading angle=180]
    (O) ++ (180+#1:\Ra) to[out={-90+#1},in={180+#1},looseness=.8]
    ($(CS)+(-90+#1:\Rb)$) arc (-90+#1:90+#1:\Rb) to[out=180+#1,in=90+#1,looseness=.8] cycle;
  
  % ROD
  \draw[thick,mydarkblue,top color=blue!30!black!50,bottom color=blue!30!black!20,shading angle=\c]
    (CS) ++ (\c-\b:\Rb) arc (\c-\b:-360+\c+\b:\Rb) -- ($(P)+(\c+90:\Rc)$) -- ($(P)+(\c-90:\Rc)$) -- cycle;
  
  % PISTON
  \draw[mydarkblue,thick,top color=blue!20!black!30,bottom color=blue!20!black!30,middle color=blue!5,shading angle=90]
    (PL) rectangle (PR);
  \draw[thick,mydarkblue,fill=knob]
    (PL) ++ (0,.65*\p) rectangle ($(PR)+(0,-.25*\p)$);
  
  % DECORATION
  \draw[thick,mydarkblue,fill=knob] (O) circle (\Rc/2);
  \draw[thick,mydarkblue,fill=knob] (CS) circle (\Rc);
  \draw[thick,mydarkblue,fill=knob] (P) circle (\Rc);
  
  % WALL
  \wall
}

% WALL
\def\wall{
  \draw[line width=4,blue!10!black!50]
    (VRo) ++ (1.5,0.6) to[out=180,in=60] (VRo) -- (TR) -- %to[out=-30,in=90,looseness=0.5]
    (RR) arc (\a:-180-\a:\R) --
    (TL) -- (VLo) to[out=110,in=0] ++ (-1.5,0.6); %to[out=90,in=200,looseness=0.8]
  \draw[line width=4,blue!10!black!50]
    (VLo) ++ (-1.5,1.3) to[out=0,in=110] (VLm) -- (S) --
    (VRm) to[out=60,in=180] ($(VRo)+(1.5,1.3)$);
    
  \fill[blue!30!black!60]
    (S) ++ (.07,.2) to[out=90,in=-150]++ (1,1.4) -- ($(S)+(1,1.74)$)
    to[out=-150,in=90] ($(S)+(-.07,.2)$);
  \draw[blue!30!black!80]
    (S) ++ (.07,.2) to[out=90,in=-150]++ (1,1.4)
    (S) ++ (-.07,.2) to[out=90,in=-150]++ (1.07,1.54);
  \draw[blue!10!black,fill=blue!20!black]
    (S) ++ (-.09,-.16) --++ (.09,-.1) coordinate (X) --++ (.09,.1) -- cycle;
  \draw[blue!30!black,fill=blue!30!black!80]
    (S) ++ (-.1,-.15) --++ (.2,0) --++ (0,.35) --++ (-.2,0) -- cycle;
}

% VALVE
\def\valveL#1{
  \fill[thick,blue!20!black]
    (VLo) ++ (\vl-90:#1) -- ($(VLm)+(\vl-90:#1)$) -- ($(VLo)!.64!(VLm)+(\vl+90:.2-#1)$) --++ (\vl+90:2)
    -- ($(VLo)!.36!(VLm)+(\vl+90:2.2-#1)$) -- ($(VLo)!.36!(VLm)+(\vl+90:.2-#1)$) -- cycle;
}

% VALVE
\def\valveR#1{
  \fill[thick,blue!20!black]
    (VRo) ++ (\vr+90:#1) -- ($(VRm)+(\vr+90:#1)$) -- ($(VRo)!.64!(VRm)+(\vr-90:.2-#1)$) --++ (\vr-90:2)
    -- ($(VRo)!.36!(VRm)+(\vr-90:2.2-#1)$) -- ($(VRo)!.36!(VRm)+(\vr-90:.2-#1)$) -- cycle;
}

\section{Les groupes motopropulseurs}

	\subsection{Moteurs à pistons}
		\subsubsection{Description du moteur à piston}
		Le moteur à piston, également appelé moteur à combustion interne ou moteur à explosion est un moteur qui transforme l'énergie chimique contenue dans le carburant (essence, gasoil, gaz...) en énergie mécanique.
		\begin{figure}[H]
  		\centering
    		% INTAKE STROKE
\begin{tikzpicture}[scale=\echelleTikz]
  \def\d{-60}
  \engine{10};
  \draw[vector] (\d:.6*\R) arc (\d:\d-80:.55*\R);
  \fill[\gas]
    (VLo) to[out=110,in=0] ++ (-1.5,0.6) -- ($(VLo)+(-1.5,1.3)$) to[out=0,in=110] (VLm) to[out=\vl-90,in=\vl-90] cycle;
  \wall
  \valveL{.3}
  \valveR{.1}
  
  \draw[arrow] (VL) ++ (-.2,.2) --++ (-1,-.5)
    node[below left=-2,align=right,scale=1.4] {valve\\[-2pt]d'admission};
  \draw[arrow] (VR) ++ (.2,.1) --++ (1,-.5)
    node[below right=-2,align=left,scale=1.4] {valve\\[-2pt]d'échappement};
  \draw[arrow] (O) ++ (-.2,.4) --++ (-2.0,.9)
    node[left=-20,above left=2,scale=1.4] {villebrequin};
  \draw[arrow] (P) ++ (1.1,-.2) --++ (1.2,-.5)
    node[below right=-2,scale=1.4] {piston};
  \draw[arrow] (P) ++ (1.5,0.4) --++ (1.2,-.5)
    node[below right=-2,scale=1.4] {cylindre};
  \draw[arrow] (P) ++ (-1,0.9) --++ (-1.2,-.5)
    node[below left=-2,scale=1.4] {segment};
  \draw[arrow] (PL) ++ (2.2,-1.9) --++ (1.68,-.7)
    node[below right=-2,scale=1.4] {bielle};
  \draw[arrow] (O) ++ (-1.9,-0.8) --++ (-1.2,.5)
    node[left=-20,above left=2,scale=1.4] {carter};
  \draw[arrow] (S) ++ (150:.2) --++ (-.5,.7)
    node[above=-1,align=center,scale=1.4] {bougie};
  \draw[arrow] (VLm) ++ (-1.5,.5) --++ (-1.5,.7)
    node[above left=-2,align=right,scale=1.4] {pipe\\[-2pt]d'admission};
  \draw[arrow] (VRm) ++ (1.5,.5) --++ (1.5,.7)
    node[above right=-2,align=left,scale=1.4] {pipe\\[-2pt]d'échappement};
  
\end{tikzpicture}
  		\caption{Schéma d'un moteur à piston (\cite{tikz::schemaMoteurPiston})}
		\end{figure}
		
		Le moteur à piston est composé des éléments principaux suivants :
		\begin{itemize}
			\item cylindre
			\item piston : pièce mobile dans le cylindre
			\item bielle : pièce qui fait le jonction entre le cylindre et le vilebrequin
			\item vilebrequin : manivelle qui converti le mouvement alternatif du piston en mouvement rotatif
			\item bougie : système d'allumage qui permet de commander la combustion du mélange contenu dans le cylindre
			\item soupape d'admission et d'échappement : pièces mobiles qui permettent de faire rentrer le mélange et de faire sortir les gaz d'échappement du cylindre 
			\item pipe d'admission et d'échappement : tubes qui permettent d'acheminer le mélange air-essence dans le réservoir et d'acheminer les gaz brulés vers l'échappement
			\item carter : bas du moteur. Contient notamment l'huile nécessaire au fonctionnement du moteur
			\item segment : anneau métallique installé sur le cylindre. Assure l'étanchéité entre le piston et le cylindre.
		\end{itemize}
	
		\subsubsection{Le cycle à 4 temps}
		\renewcommand{\echelleTikz}{0.5}
		\paragraph{Admission}
		
		L'admission est le premier cycle du cycle à 4 temps. Durant cette phase, qui démarre alors que le piston est en point haut, la soupape d'admission s'ouvre. La descente du piston durant cette étape permet l'aspiration du mélange air-carburant dans le cylindre. Lorsque le cylindre atteint son point bas, la soupape d'admission est refermée.

		\begin{figure}[H]
  		\centering
    		% INTAKE STROKE
\begin{tikzpicture}[scale=\echelleTikz]
  \def\d{-60}
  \engine{10};
  \draw[vector] (\d:.6*\R) arc (\d:\d-80:.55*\R);
  \fill[\gas]
    (VLo) to[out=110,in=0] ++ (-1.5,0.6) -- ($(VLo)+(-1.5,1.3)$) to[out=0,in=110] (VLm) to[out=\vl-90,in=\vl-90] cycle;
  \wall
  \valveL{.3}
  \valveR{.1};
  
\end{tikzpicture}
  		\caption{Étape 1 : Admission}
		\end{figure}	
	
		\paragraph{Compression}
		
		Dans ce deuxième cycle, qui débute alors que le piston est en position basse, les 2 soupapes sont fermées. Le piston remonte et le mélange air-carburant précédemment admis est comprimé dans le cylindre.
		
		\begin{figure}[H]
  		\centering
    		% COMPRESSION STROKE
\begin{tikzpicture}[scale=\echelleTikz]
  \def\d{-10}
  \engine{-140};
  \draw[vector] (\d:.6*\R) arc (\d:\d-80:.55*\R);
  \valveL{.1}
  \valveR{.1}
\end{tikzpicture}
  		\caption{Étape 2 : Compression}
		\end{figure}	
		
		\paragraph{Explosion-détente}
		
		Dans ce cycle, qui démarre alors que le piston atteint à nouveau le point haut, l'étincelle provoquée par la bougie provoque l'explosion du mélange présent dans le cylindre. Le piston est alors repoussé vers le bas. Durant ce cycle, les 2 soupapes restent fermées.
		
		\begin{figure}[H]
  		\centering
		% IGNITION
\begin{tikzpicture}[scale=\echelleTikz]
  \def\d{40}
  \engine{90};
  \draw[vector] (\d:.6*\R) arc (\d:\d-80:.55*\R);
  \valveL{.1}
  \valveR{.1}
  \draw[very thin,yellow!70!black,fill=yellow,shift={(X)}]
    ( -15:.20) -- ( -30:.40) -- ( -40:.25) -- ( -50:.40) --
    ( -60:.22) -- ( -70:.40) -- ( -80:.20) -- ( -90:.45) --
    (-100:.24) -- (-110:.40) -- (-120:.25) -- (-130:.40) --
    (-140:.20) -- (-150:.45) -- (-165:.20) to[out=40,in=140] cycle;
\end{tikzpicture}
  		\caption{Étape 3 : Explosion-détente}
		\end{figure}	
		
		\info{Le cycle d'explosion-détente est le seul cycle qui produit effectivement de l'énergie.}
		
		\paragraph{Échappement}
		
		Dans cette quatrième et dernière étape du cycle, la soupape d'échappement est ouverte. Le piston, initialement en point bas, remonte et pousse les gaz brûlés issues de la combustion en dehors du cylindre lors de la remontée. L'étape d'échappement se termine lorsque le piston atteint le point haut, la soupape d'échappement est alors refermée.
		
		\begin{figure}[H]
  		\centering
		% EXHAUST STROKE
\begin{tikzpicture}[scale=\echelleTikz]
  \def\d{-40}
  \engine{-190};
  \draw[vector] (\d:.6*\R) arc (\d:\d-80:.55*\R);
  \fill[\gas]
    (VRo) to[out=60,in=-180] ++ (1.5,0.6) -- ($(VRo)+(1.5,1.3)$) to[out=180,in=60] (VRm) to[out=-270+\vr,in=-270+\vr] cycle;
  \wall
  \valveL{.1}
  \valveR{.3}
\end{tikzpicture}
  		\caption{Étape 4 : Échappement}
		\end{figure}	
		
		\paragraph{Cycle complet}
		
		L'animation suivante permet de visualiser le fonctionnement d'un moteur à explosion.	
		
		\renewcommand{\echelleTikz}{0.5}
		\begin{figure}[H]
  		\centering
		%\begin{animateinline}[autoplay,loop,controls]{60}
 \multiframe{360}{i=-90+2}{
    \begin{tikzpicture}
	  \coordinate (boiteLegende1) at (0,8.5);
    	  \coordinate (boiteLegende2) at (0,9);
      \node[rectangle,minimum width=2cm] [fit = (boiteLegende1) (boiteLegende2)] (legende) {};    
    
      %\def\d{-10}
      \engine{-\i};
      %\draw[vector] (\d:.6*\R) arc (\d:\d-80:.55*\R);
      %\valveL{.1}
      %\valveR{.1}

	% INTAKE STROKE
	\ifnum\i>-92 \ifnum\i<-88 
	    \node[align=center,font=\Large] at (legende.center) {\textbf{Admission}};
		\valveL{.2}
  		\valveR{.1} ;
	\fi \fi 
	\ifnum\i>-90 \ifnum\i<90 
	    \node[align=center,font=\Large] at (legende.center) {\textbf{Admission}};
		\fill[\gas]
    		(VLo) to[out=110,in=0] ++ (-1.5,0.6) -- ($(VLo)+(-1.5,1.3)$) to[out=0,in=110] (VLm) to[out=\vl-90,in=\vl-90] cycle;
  		\wall
		\valveL{.3}
  		\valveR{.1} ;
	\fi \fi
	\ifnum\i>88 \ifnum\i<92 
	    \node[align=center,font=\Large] at (legende.center) {\textbf{Admission}};
		\valveL{.2}
  		\valveR{.1} ;
	\fi \fi 
	% COMPRESSION STROKE
	\ifnum\i>90 \ifnum\i<272 
	    \node[align=center,font=\Large] at (legende.center) {\textbf{Compression}};
		\valveL{.1}
  		\valveR{.1} ;
	\fi \fi 
	% POWER STROKE
	\ifnum\i>270 \ifnum\i<452 
		\node[align=center,font=\Large] at (legende.center) {\textbf{Explosion-détente}};
		\valveL{.1}
  		\valveR{.1} ;
	\fi \fi
	% EXHAUST STROKE
	\ifnum\i>450 \ifnum\i<454
	    \node[align=center,font=\Large] at (legende.center) {\textbf{Echappement}};
		\valveL{.1}
  		\valveR{.2} ;
	\fi \fi
	\ifnum\i>452 \ifnum\i<630
	    \node[align=center,font=\Large] at (legende.center) {\textbf{Echappement}};
		\fill[\gas]
    		(VRo) to[out=60,in=-180] ++ (1.5,0.6) -- ($(VRo)+(1.5,1.3)$) to[out=180,in=60] (VRm) to[out=-270+\vr,in=-270+\vr] cycle;
  		\wall 
		\valveL{.1}
  		\valveR{.3} ;
	\fi \fi
	\ifnum\i>630 \ifnum\i<632
	    \node[align=center,font=\Large] at (legende.center) {\textbf{Echappement}};
		\valveL{.1}
  		\valveR{.2} ;
	\fi \fi

	% IGNITION
	\ifnum\i>268 \ifnum\i<280 
		\draw[very thin,yellow!70!black,fill=yellow,shift={(X)}]
    		( -15:.20) -- ( -30:.40) -- ( -40:.25) -- ( -50:.40) --
    		( -60:.22) -- ( -70:.40) -- ( -80:.20) -- ( -90:.45) --
    		(-100:.24) -- (-110:.40) -- (-120:.25) -- (-130:.40) --
   	 	(-140:.20) -- (-150:.45) -- (-165:.20) to[out=40,in=140] cycle; 
	\fi \fi 

      
    \end{tikzpicture}
  }
\end{animateinline}
  		\caption{Le fonctionnement d'un moteur à explosion (\cite{tikz::schemaMoteurPistonAnime})}
		\end{figure}	
	
		
	
	\subsection{Motorisation électrique}
	
	\subsection{Turbopropulseurs et turbomoteurs}
	
	\subsection{Hélices et moteurs}
	
	\subsection{Propulseurs à réaction}
		\subsubsection{Turboréacteurs}
	
		\subsubsection{Statoréacteurs}
	
		\subsubsection{Moteurs fusées}
		
	\subsection{Contraintes liées au développement durable}
		\section{Structures et matériaux}
		\section{Les commandes de vol}
		\section{L'instrumentation de bord}
	
	\chapter{Navigation, réglementation, sécurité des vols\label{nav}}
		\section{Navigation}
	\subsection{Les grands principes de navigation}
	
	\subsection{Les outils de navigation}
		\subsubsection{Cartes aéronautiques}
		
		\subsubsection{Aides à la navigation}
			\paragraph{Notions de base}
			
			\paragraph{Le VOR}
			
			\subparagraph{Le VOR-DME}
			
			\paragraph{L'ADF}
			
			\paragraph{Le GPS}
		\section{Réglementation aéronautique}
	\subsection{L'organisation de l'espace aérien}
	
	\subsection{Titres aéronautiques}
	
	\subsection{Les organisations}
	
	\subsection{Contrôle d'un aéronef}
		\section{Sécurité des vols}
	\subsection{Gestion des risques}
	
	\subsection{Performances humaines et ses limites}
	
	\subsection{Prise de décision}
	
	\chapter{Météorologie et aérologie}
	\label{meteo}
		\section{L'atmospère}
		\section{Les nuages}
		\section{Les vents}
		\section{Les masses d'air et les fronts}
		\section{Les phénomènes dangereux pour le vol}
	
	\chapter{Aérodynamique, aérostatique et principes du vol}
	\label{aerodynamique}
		\section{La sustentation et l'aile}



		\section{Étude du vol stabilisé}
	\subsection{Vol plané}
	
	\subsection{Vol stabilisé}
		\section{L'aérostation}
	\subsection{Principe généraux}
		\section{Le vol spatial}
	\subsection{Principe généraux}
	
	\chapter{Histoire et culture de l'aéronautique et du spatial}
	\label{histoire}
		\section{Du mythe à la réalité}
		\section{Des précurseurs aux pionniers}
		\section{Les enjeux militaires et les évolutions de l'aéronautique et du spatial}
		\section{Les enjeux économiques et les évolutions de l'aéronautique et du spatial}

	
%\chapter{Table des matières}
	\setcounter{tocdepth}{5}
	\tableofcontents
	
	\listoffigures
	
	\printglossaries

%\chapter{Bibliographie}
	\nocite{*}
	\bibliographystyle{unsrt-fr}
	\bibliography{biblio}

	
	
	
			
	
\end{document}