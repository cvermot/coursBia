\section{Réglementation aéronautique}
	\subsection{L'organisation de l'espace aérien}
	Afin de permettre aux aéronefs d'évoluer en toute sécurité, l'espace aérien a été divisé en différents espaces, qui présentent chacun des conditions d'accès spécifiques et des services associés.
		
		\subsubsection{Les classes d'espaces aériens}
		Au niveau mondial, l'OACI à définit 7 classes d'espaces aériens, nommées par des lettres de A (classe présentant le plus de contraintes) à G (classe présentant le moins de contraintes). Dans ce chapitre, nous allons voir quelles sont les différences entre ces classes.
		
		\subsubsection{Les zones restreintes}
			\paragraph{Les zones R}
			Les zones R sont des zones \textbf{R}estreintes. La pénétration de ces zones est possible sous condition. Par exemple, \hyperlink{ignOaciBordeaux.1}{sur la carte extrait de Bordeaux (cf \ref{ignOaciBdx} page \pageref{ignOaciBdx})}, on peut entrer dans la zone R~204~L3 si on est en contact avec le service d'information de vol Aquitaine Info\footcite{Source : exterait de l'ENR5.1 "VFR : pénétration après contact radio avec AQUITAINE INFO. Veille radio obligatoire."}
			
			\paragraph{Les zones D}
			Les zone D sont des zones \textbf{D}angereuses.
			
			\paragraph{Les zones P}
			Les zones P sont des zones interdites (\textbf{P}rohibées).
	
	\subsection{Titres aéronautiques}
	
	\subsection{Les organisations}
	
	\subsection{Contrôle d'un aéronef}