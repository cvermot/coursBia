\section{Réglementation aéronautique}
	\subsection{L'organisation de l'espace aérien}
	Afin de permettre aux aéronefs d'évoluer en toute sécurité, l'espace aérien a été divisé en différents espaces, qui présentent chacun des conditions d'accès spécifiques et des services associés.
		
		\subsubsection{Les classes d'espaces aériens}
		Au niveau mondial, l'OACI à définit 7 classes d'espaces aériens, nommées par des lettres de A (classe présentant le plus de contraintes) à G (classe présentant le moins de contraintes). Dans ce chapitre, nous allons voir quelles sont les différences entre ces classes.
		
		\paragraph{Classe A}
		
		\paragraph{Classe B}
		
		\paragraph{Classe C}
		
		\paragraph{Classe D}\label{classeD}
		
		\paragraph{Classe E}
		
		\paragraph{Classe F}
		
		\paragraph{Classe G}
		
		\subsubsection{Les zones à statut particulier}
			\paragraph{Les zones R}
			Les zones R sont des zones \textbf{R}estreintes. La pénétration de ces zones est possible sous condition. Par exemple, \hyperlink{ignOaciBordeaux.1}{sur la carte extrait de Bordeaux (cf \ref{ignOaciBdx} page \pageref{ignOaciBdx})}, on peut entrer dans la zone R~204~L3 si on est en contact avec le service d'information de vol Aquitaine Info\footnote{Source : extrait de l'ENR5.1 "VFR : pénétration après contact radio avec AQUITAINE INFO. Veille radio obligatoire."}.
			
			Ces zones peuvent être actives en permanence ou seulement sur certaines tranches horaire.
			
			\paragraph{Les zones D}
			Les zone D sont des zones \textbf{D}angereuses. Leur pénétration n'est pas formellement interdite mais les activités qui s'y déroulent sont suffisamment risquées pour qu'il existe une nécessité d'avertir les navigateurs aériens. Pour ces zones, on dispose généralement d'un organisme à contacter par radio pour connaitre l'activité réelle.
			
			\alert{Il ne faut pas confondre les zones D avec les espaces aériens de classe D (cf \ref{classeD} page \pageref{classeD}}.
			
			Ces zones peuvent être actives en permanence ou seulement sur certaines tranches horaire.
			
			\paragraph{Les zones P}
			Les zones P sont des zones interdites (\textbf{P}rohibées). Le vol dans ces zones est en général strictement interdit. En France, ces zones protègent généralement des infrastructures critiques, comme des centrales nucléaires. Par exemple, \hyperlink{ignOaciBordeaux.1}{sur la carte extrait de Bordeaux (cf \ref{ignOaciBdx} page \pageref{ignOaciBdx})}, la zone P~5 protège les installations du laser mégajoule au Barp. Cette zone est interdit de pénétration sauf pour les vols IFR ayant obtenu l'autorisation du contrôle, les vols de secours qui ne peuvent contourner dans le cadre de leur missions, et les aéronefs spécifiquement autorisés\footnote{Source : ENR 5.1 1.2.1}.
			
			Ces zones sont généralement actives en permanence.
	
	\subsection{Titres aéronautiques}
	
	\subsection{Les organisations}
	
	\subsection{Contrôle d'un aéronef}