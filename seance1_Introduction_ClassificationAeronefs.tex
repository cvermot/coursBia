\ifdefined\activerhandout 
\documentclass[12pt,aspectratio=1610,handout]{beamer}
\else
\documentclass[12pt,aspectratio=1610]{beamer}
\fi

%\usepackage[french]{babel} %=> erreur avec les tikz du moteur
\usepackage[T1]{fontenc}
\usepackage[utf8]{inputenc}
\usepackage{lmodern}
\usepackage{hyperref}
\usepackage{smartdiagram}
\usepackage{tikz}
\usepackage{animate}
\usepackage{tikzpeople}
\usepackage{appendixnumberbeamer}
\usepackage[labelformat=empty]{caption}
%\usepackage{pictochrono}
\usepackage{fontawesome5}
\usepackage{awesomebox}

\usetheme{Warsaw}
\setbeamertemplate{page number in head/foot}[totalframenumber]

\newcommand{\anglais}[1]{(\textit{\color{blue}#1})}
\newcommand{\legende}[2]{\caption[#1 (Source : \cite{#2})]{#1}}
\newcommand{\histoire}[1]{\begin{awesomeblock}{2pt}{\faBook}{black!75}#1\end{awesomeblock}}
\newcommand{\info}[1]{\begin{awesomeblock}{2pt}{\faInfoCircle}{black!75}#1\end{awesomeblock}}
\newcommand{\question}[1]{\begin{awesomeblock}{2pt}{\faQuestionCircle}{black!75}#1\end{awesomeblock}}
\newcommand{\alerte}[1]{\begin{awesomeblock}{2pt}{\faExclamationCircle}{black!75}#1\end{awesomeblock}}
\newcommand{\astuce}[1]{\begin{awesomeblock}{2pt}{\faLightbulb}{black!75}#1\end{awesomeblock}}
\newcommand{\exemple}[1]{\begin{awesomeblock}{2pt}{\faSearch}{black!75}#1\end{awesomeblock}}
\newcommand{\definitionAConnaitre}[1]{\begin{awesomeblock}{2pt}{\faCog}{black!75}#1\end{awesomeblock}}

\newcommand{\qmcBia}[7]{
%1 : titre slide
%2 : numéro de la bonne réponse
%3 : Inititulé de la question
%4, 5, 6, et 7 : propositions de réponse
\begin{frame}{#1}
\begin{awesomeblock}{2pt}{\faQuestion}{black!75}
#3
	\begin{enumerate}
	\ifnum#2=1
		\only<1>{\item #4}
		\only<2>{\item \textbf{#4}}
	\else
		\item #4
	\fi
	\ifnum#2=2
		\only<1>{\item #5}
		\only<2>{\item \textbf{#5}}
	\else
		\item #5
	\fi
	\ifnum#2=3
		\only<1>{\item #6}
		\only<2>{\item \textbf{#6}}
	\else
		\item #6
	\fi
	\ifnum#2=4
		\only<1>{\item #7}
		\only<2>{\item \textbf{#7}}
	\else
		\item #7
	\fi
	\end{enumerate}
	\pause
\end{awesomeblock}
\end{frame}
}

\subtitle{BIA - Brevet d'Initiation Aéronautique}
\author{Clément \textsc{Vermot-Desroches}}
\institute{Collège Aliénor d'Aquitaine\\Martignas-sur-Jalle}
\date{\today}

%\AtBeginSection[]
%{
%    \begin{frame}
%        %\frametitle{Table of Contents}
%        \tableofcontents[currentsection]
%    \end{frame}
%}

\AtBeginSubsection[]
{
    \begin{frame}
        %\frametitle{Table of Contents}
        \tableofcontents[currentsection,currentsubsection]
    \end{frame}
}

\addbibresource{01-EtudeAeronefs/biblio.bib}
\addbibresource{05-Histoire/biblio.bib}

\title[Séance 1 - Introduction]{Séance 1 \\ Introduction \& Classification des aéronefs}

\begin{document}
 \begin{frame}
 \titlepage
 \end{frame}
 
 \begin{frame}
 \tableofcontents
 \end{frame}
 
\section{Introduction}
\begin{frame}{Qui sommes nous ? Tour de table}

\begin{awesomeblock}{2pt}{\faChild}{black!75}
Prénom

Nom
\end{awesomeblock}	

\begin{awesomeblock}{2pt}{\faPlane}{black!75}
Qu'est ce que je connais de l'aéronautique ?
\end{awesomeblock}

\begin{awesomeblock}{2pt}{\faQuestion}{black!75}
Pourquoi suis-je la ?
\end{awesomeblock}

\end{frame}

\begin{frame}{Ce que j'attends de vous cette année}
\begin{itemize}
  \item De la curiosité
  \item N'ayez pas peur de vous tromper !
  \item Pas de travail personnel obligatoire...
  \item ... mais si vous travaillez un peu de votre côté, vous augmentez vos chances d'obtenir le BIA et de l'avoir avec une mention
  \begin{itemize}
  	\item si vous avez des questions suite à des choses que vous aurez étudié de votre côté : posez les !
  \end{itemize}
\end{itemize}

Programme dense car nous démarrons en cours d'année

\importantbox{N'hésitez pas à poser toutes vos questions pendant le cours}
\end{frame}


\section{Classification des aéronefs}
\begin{frame}{Activité de groupe}

\begin{awesomeblock}{2pt}{\faQuestion}{black!75}
Pouvez vous me citer un maximum de machines volantes ?
\end{awesomeblock}
\pause 
\begin{awesomeblock}{2pt}{\faQuestion}{black!75}
Comment peut-on regrouper ces différentes machines ?
\end{awesomeblock}	
\end{frame}




\subsection{Pourquoi classer les aéronefs ?}
\begin{frame}{Pourquoi classer les aéronefs ? - Définitions}

Chaque type d'aéronef dispose de propriétés et de contraintes qui lui sont propres. 

La classification des aéronefs en grand groupes présentant des caractéristiques communes permet de leur associer aisément des notions réglementaires (licence de pilote nécessaire, minima météo, zone de vol autorisées...), techniques (fréquences et mode d'entretien, contraintes de conception) ou administratives (immatriculation, assurance...).
\pause
Un aéronef \anglais{aircraft} est un appareil capable de s'élever et de se mouvoir au sein de l'atmosphère terrestre. \pause On divise les aéronefs en 2 grandes familles : \pause
\begin{itemize}
	\item les  aérostats \anglais{aerostat/lighter-than-air aircraft}, qui sont des appareils plus légers que l'air,
	\pause
	\item les  aérodynes \anglais{heavier-than-air aircraft}, qui sont plus lourds que l'air.
\end{itemize}
\end{frame}

\subsection{Les grands types d'aéronefs}

\begin{frame}{Aérostats - la montgolfière \anglais{hot air balloon}}
\begin{columns}
\begin{column}{0.65\textwidth}

\begin{itemize}
	\item aérostat gonflé à l'air chaud
	\item composé d'un ballon (enveloppe) sous lequel est accroché une nacelle
	\item un bruleur permet de chauffer l'air du ballon
	\item déplacement soumis au vent
\end{itemize}

\only<2>{\histoire{Invention française des frères Montgolfier

1783 : premier vol : un coq, un mouton et un canard puis Pilâtre de Rosier

1785 : Jean-Pierre Blanchard traverse la Manche en ballon}}

\end{column}
\begin{column}{0.35\textwidth}	
	\begin{figure}[H]
  	\centering
    \includegraphics[width=0.95\textwidth]{01-EtudeAeronefs/img/montgolfiere.jpg}
  	\legende{2 ballons}{img:montgolfiere}
	\end{figure}	
\end{column}
\end{columns}
\pause 
\end{frame}


\begin{frame}{Aérostats - le ballon à gaz \anglais{gas balloon}}
\begin{columns}
\begin{column}{0.65\textwidth}

\begin{itemize}
	\item ballon est gonflé avec un gaz plus léger que l'air (hydrogène ou hélium)
	\item autonomie de vol importante
\end{itemize}

\only<2>{\histoire{Invention du français Jean Charles, en 1783}}

\end{column}
\begin{column}{0.35\textwidth}	
	\begin{figure}[H]
  	\centering
    \includegraphics[width=0.80\textwidth]{01-EtudeAeronefs/img/ballonAGaz.jpg}
  	\legende{Un ballon à l'hélium moderne}{img:ballonAGaz}
	\end{figure}	
\end{column}
\end{columns}
\pause 
\end{frame}
	
\begin{frame}{Aérostats - le dirigeable \anglais{airship}}

\begin{columns}
\begin{column}{0.65\textwidth}

\begin{itemize}
	\item ballon à gaz équipé d'un système propulsif
	\item gonflé à l'hydrogène puis l'hélium
	\item corrige le principale défaut des ballons : leur non dirigeabilité
\end{itemize}

\end{column}
\begin{column}{0.35\textwidth}	
	\begin{figure}[H]
  	\centering
    \includegraphics[width=0.95\textwidth]{01-EtudeAeronefs/img/dirigeable.jpg}
  	\legende{Un dirigeable moderne}{img:dirigeable}
	\end{figure}	
\end{column}
\end{columns}
\pause 
{\histoire{L'ingénieur français Henri Giffard fit voler le premier dirigeable en 1852.}}
\end{frame}

\begin{frame}{Aérodynes - Voilure fixe - Avion \anglais{airplane}}
\begin{columns}
\begin{column}{0.65\textwidth}
\begin{itemize}
	\item la portance est assurée par le flux d'air sur les surfaces portantes de l'avion (aile \anglais{wing})
	\item la portance n'existe qu'avec une vitesse de l'air suffisante
	\item sur un avion, la mise en mouvement est assurée par un ou plusieurs moteurs 
\end{itemize}

\end{column}
\begin{column}{0.35\textwidth}	
	\begin{figure}[H]
  	\centering
    \includegraphics[width=0.95\textwidth]{01-EtudeAeronefs/img/cessna172.jpg}
  	\legende{Un avion Cessna 172}{img:cessna172}
	\end{figure}	
\end{column}
\end{columns}
\pause 

{\histoire{Le premier vol d'un avion a été réalisé en 1903 par les frères Wright, aux États-Unis}}
\end{frame}

\begin{frame}{Aérodynes - Voilure fixe - Planeur \anglais{glider}}
\begin{columns}
\begin{column}{0.65\textwidth}
\begin{itemize}
	\item aérodyne dépourvu de moyens de propulsion
	\item mise en l'air grâce à un remorqueur ou un treuil
	\item une fois en vol peut monter grâce aux courants atmosphériques
\end{itemize}

\end{column}
\begin{column}{0.35\textwidth}	
	\begin{figure}[H]
  	\centering
    \includegraphics[width=0.95\textwidth]{01-EtudeAeronefs/img/planeur.jpg}
  	\legende{Un planeur}{img:planeur}
	\end{figure}	
\end{column}
\end{columns}
\pause

{\histoire{Premier vol d'un planeur en 1891 par l'allemand \textbf{Otto Lilienthal}, en partant d'une colline}} 
\end{frame}

\begin{frame}{Aérodynes - Voilures tournantes - Hélicoptère \anglais{helicopter}}
\begin{columns}
\begin{column}{0.65\textwidth}
\begin{itemize}
	\item vol assuré exclusivement par un rotor entrainé directement par un moteur
	\item présence d'un rotor anticouple
\end{itemize}

\end{column}
\begin{column}{0.35\textwidth}	
	\begin{figure}[H]
  	\centering
    \includegraphics[width=0.95\textwidth]{01-EtudeAeronefs/img/helicoptereEC145.jpg}
  	\legende{Un hélicoptère EC145}{img:helicoptereEC145}
	\end{figure}	
\end{column}
\end{columns}
\pause 

{\histoire{1936 : premier vol réellement contrôlé pour un hélicoptère (Focke-Wulf Fw 61)}}
\end{frame}

\begin{frame}{Aérodynes - Voilures tournantes - Autogire \anglais{autogyro}}
\begin{columns}
\begin{column}{0.65\textwidth}
\begin{itemize}
	\item Portance assurée par un rotor entrainé par le vent relatif
	\item Présence d'un système de propulsion
\end{itemize}

\end{column}
\begin{column}{0.35\textwidth}	
	\begin{figure}[H]
  	\centering
    \includegraphics[width=0.95\textwidth]{05-Histoire/img/CiervaC6.jpg}
  	\legende{Le C6 de Juan de la Cierva (1924)}{img:CiervaC6}
	\end{figure}	
\end{column}
\end{columns}
\pause
{\histoire{Premier vol d'un autogire en 1923 par l'espagnol Juan de la Cierva.}}
\end{frame}

\begin{frame}{}
\begin{columns}
\begin{column}{0.65\textwidth}
\begin{itemize}
	\item 
	\item 
	\item 
	\item 
\end{itemize}

\only<2>{\histoire{}}

\appendix
\section{QCM}
\qmcBia{Étude des aéronefs}
{1}{Tout appareil capable de s'élever et de circuler dans l'espace aérien :}
{est un aéronef}
{subit des forces aérodynamiques}
{possède obligatoirement un moteur}
{est piloté depuis l'intérieur de son cockpit}

\qmcBia{Étude des aéronefs}
{3}{Un aéronef qui, en croisière, voit son rotor entraîné par le vent relatif est :}
{un convertible}
{un girodyne}
{un autogire}
{un hélicoptère}

\qmcBia{Histoire}
{2}{En 1783, le premier vol d’un ballon à air chaud est rendu possible grâce au travail des frères :}
{Wright}
{Montgolfier}
{Caudron}
{Voisin}

\qmcBia{Histoire}
{1}{La première traversée de la Manche avec un aéronef a été réalisée :}
{en 1785 par Jean-Pierre Blanchard et John Jeffries}
{en 1852 par Henry Giffard}
{en 1901 par Alberto Santos-Dumont}
{en 1909 par Louis Blériot}

\qmcBia{Anglais}
{3}{Le terme anglais "airship" désigne principalement :}
{tout type d’aéronef}
{les planeurs}
{les ballons dirigeables}
{les avions gros porteurs}

\ifdefined\activerbibliobeamer
\begin{frame}[allowframebreaks]
\frametitle{Bibliographie}
\printbibliography
%\nocite{*}
\end{frame}
\fi 

\end{document}

