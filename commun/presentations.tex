\ifdefined\activerhandout 
\documentclass[12pt,aspectratio=1610,handout]{beamer}
\else
\documentclass[12pt,aspectratio=1610]{beamer}
\fi

%\usepackage[french]{babel} %=> erreur avec les tikz du moteur
\usepackage[T1]{fontenc}
\usepackage[utf8]{inputenc}
\usepackage{lmodern}
\usepackage{hyperref}
\usepackage{smartdiagram}
\usepackage{xcolor}
\usepackage{tikz}
\usepackage{animate}
\usepackage{tikzpeople}
\usepackage{appendixnumberbeamer}
\usepackage[labelformat=empty]{caption}
%\usepackage{pictochrono}
\usepackage{fontawesome5}
\usepackage{awesomebox}

\usepackage[
    backend=biber,
    %style=authoryear-icomp,
    style=numeric,
    %style=ieee,
    sortlocale=fr_FR,
    natbib=true,
    url=true, 
    doi=true,
    eprint=true
]{biblatex}

\usetheme{Warsaw}
\setbeamertemplate{page number in head/foot}[totalframenumber]

\newcommand{\anglais}[1]{(\textit{\color{blue}#1})}
\newcommand{\legende}[2]{\caption{#1\ifdefined\activerbibliobeamer \cite{#2}\fi}}
\newcommand{\histoire}[1]{\begin{awesomeblock}{2pt}{\faBook}{black!75}#1\end{awesomeblock}}
\newcommand{\info}[1]{\begin{awesomeblock}{2pt}{\faInfoCircle}{black!75}#1\end{awesomeblock}}
\newcommand{\question}[1]{\begin{awesomeblock}{2pt}{\faQuestionCircle}{black!75}#1\end{awesomeblock}}
\newcommand{\alerte}[1]{\begin{awesomeblock}{2pt}{\faExclamationCircle}{black!75}#1\end{awesomeblock}}
\newcommand{\astuce}[1]{\begin{awesomeblock}{2pt}{\faLightbulb}{black!75}#1\end{awesomeblock}}
\newcommand{\exemple}[1]{\begin{awesomeblock}{2pt}{\faSearch}{black!75}#1\end{awesomeblock}}
\newcommand{\definitionAConnaitre}[1]{\begin{awesomeblock}{2pt}{\faCog}{black!75}#1\end{awesomeblock}}

%\newcommand{\qmcBia}[8]{
%%1 : titre slide
%%2 : numéro de la bonne réponse
%%3 : Inititulé de la question
%%4, 5, 6, et 7 : propositions de réponse
%%8 : explication
%\begin{frame}{#1}
%\begin{awesomeblock}{2pt}{\faQuestion}{black!75}
%#3
%	\begin{enumerate}
%	\ifnum#2=1
%		\only<1>{\item #4}
%		\only<2>{\item \textbf{#4}}
%	\else
%		\item #4
%	\fi
%	\ifnum#2=2
%		\only<1>{\item #5}
%		\only<2>{\item \textbf{#5}}
%	\else
%		\item #5
%	\fi
%	\ifnum#2=3
%		\only<1>{\item #6}
%		\only<2>{\item \textbf{#6}}
%	\else
%		\item #6
%	\fi
%	\ifnum#2=4
%		\only<1>{\item #7}
%		\only<2>{\item \textbf{#7}}
%	\else
%		\item #7
%	\fi
%	\end{enumerate}
%	\pause
%\end{awesomeblock}
%#8
%\end{frame}
%}

\newcommand{\qmcBia}[8]{
%1 : titre slide
%2 : numéro de la bonne réponse
%3 : Inititulé de la question
%4, 5, 6, et 7 : propositions de réponse
%8 : explication
\begin{frame}{#1}
\begin{awesomeblock}{2pt}{\faQuestion}{black!75}
#3
	\begin{enumerate}
	\item #4
	\item #5
	\item #6
	\item #7
	\end{enumerate}
\end{awesomeblock}
\end{frame}

\begin{frame}{#1}
\begin{awesomeblock}{2pt}{\faQuestion}{black!75}
#3
	\begin{enumerate}
	\ifnum#2=1
		\item \textbf{#4}
	\else
		\item #4
	\fi
	\ifnum#2=2
		\item \textbf{#5}
	\else
		\item #5
	\fi
	\ifnum#2=3
		\item \textbf{#6}
	\else
		\item #6
	\fi
	\ifnum#2=4
		\item \textbf{#7}
	\else
		\item #7
	\fi
	\end{enumerate}
\end{awesomeblock}
#8
\end{frame}
}

\subtitle{BIA - Brevet d'Initiation Aéronautique}
\author[BIA]{Clément \textsc{Vermot-Desroches}}
\institute{Collège Aliénor d'Aquitaine\\Martignas-sur-Jalle}
%\date{\today}
\date{Année 2025-2026}

\AtBeginSection[]
{
    \begin{frame}
        %\frametitle{Table of Contents}
        \tableofcontents[currentsection]
    \end{frame}
}

\AtBeginSubsection[]
{
    \begin{frame}
        %\frametitle{Table of Contents}
        \tableofcontents[currentsection,currentsubsection]
    \end{frame}
}

\addbibresource{commun/biblio.bib}
\addbibresource{01-EtudeAeronefs/biblio.bib}
\addbibresource{04-Aerodynamique/biblio.bib}
\addbibresource{05-Histoire/biblio.bib}
