%\documentclass[tikz,border=6pt]{standalone}
%\usepackage{tikz}
\usetikzlibrary{calc,backgrounds}
%\usepackage{ifthen}

%\begin{document}

\pgfdeclareimage{montagne}{commun/img/montagne.pdf}

\newcommand{\echelleAltitudeMetres}[2]{
%\begin{tikzpicture}[x=1cm,y=1cm, scale=0.6, every node/.style={scale=0.6}] % 1 unité = 1 km = 1000 m

% --------- paramètres (modifiables) ----------
\def\maxkm{#1}            % hauteur totale en km (10 = 10000 m)
\def\pointeralt{#2}      % position curseur en km (5 = 5000 m)
\def\pointerside{right}   % 'left' ou 'right' (placement du curseur)
% limites horizontales de la montagne (choisis des valeurs <= -1.6 pour ne pas empiéter sur l'axe)
\def\positionMontagne{-4.9}      % bord gauche de la base
% ---------------------------------------------

% configuration curseur selon côté
\ifthenelse{\equal{\pointerside}{right}}{%
  \def\pointerx{0.45}%
  \def\labelanchor{west}%
  \def\xshift{6pt}%
}{%
  \def\pointerx{-0.45}%
  \def\labelanchor{east}%
  \def\xshift{-6pt}%
}

% cadre optionnel pour cadrer
	\draw[gray!0] (\positionMontagne,0) rectangle (2.5,\maxkm+0.3);

% Axe vertical (altitude)
\draw[line width=0.6pt] (0,0) -- (0,\maxkm+0.3);
% graduations 1000 m (1 km)
\foreach \k in {0,...,10}{
  \pgfmathtruncatemacro{\lab}{\k*1000}
  \draw (-0.12,\k) -- (0.12,\k);
  \node[left=3pt] at (-0.15,\k) {\small \lab\ m};
}
% étiquette d'axe "Altitude" déplacée plus à gauche pour ne pas être masquée
\node[rotate=90] at (-2.0,\maxkm/2) {\large Altitude};

\node at (\positionMontagne,0) {\pgfbox[0,0]{\pgfuseimage{montagne}}};

% --- Curseur triangulaire bleu sur le côté choisi (par défaut à droite) ---
\begin{scope}
  \fill[blue!70!black] (0,\pointeralt) -- (\pointerx,\pointeralt+0.22) -- (\pointerx,\pointeralt-0.22) -- cycle;
  \draw[blue!90!black,line width=0.6pt] (0,\pointeralt) -- (\pointerx,\pointeralt+0.22) -- (\pointerx,\pointeralt-0.22) -- cycle;
  \pgfmathtruncatemacro{\pointerm}{\pointeralt*1000}
  \node[anchor=\labelanchor, xshift=\xshift] at (\pointerx,\pointeralt) {\small \pointerm\ m};
\end{scope} 
}


%\end{tikzpicture}
%\end{document}
