Dans ce chapitre, nous allons aborder l'un des composant majeur de la structure d'un aérodyne, celui qui le fait voler : son aile.

Nous décrirons dans un premier temps ce qu'est une aile et quelles sont ses caractéristiques. Nous étudierons notamment les différents types d'ailes.

Nous verrons ensuite quels sont les grands principes du vol stabilisé, ainsi qu'en montée et en descente.

Enfin, nous étudierons l'aérostatique (le vol des aérostats) et le vol spatial.