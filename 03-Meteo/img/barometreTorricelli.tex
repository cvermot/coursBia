% Author: Izaak Neutelings (November 2020)
%\documentclass[border=3pt,tikz]{standalone}
%\usepackage{siunitx}
%\usepackage{physics}
%\usepackage{tikz}
%\usepackage[outline]{contour} % glow around text

%\begin{document}

\usetikzlibrary{patterns,decorations.pathmorphing}
\usetikzlibrary{arrows.meta}
\tikzset{>=latex}
\contourlength{1.1pt}

\colorlet{mydarkblue}{blue!50!black}
\colorlet{myred}{red!65!black}
\colorlet{watercol}{blue!80!cyan!10!white}
\colorlet{darkwatercol}{blue!80!cyan!20!white}
\tikzstyle{piston}=[blue!50!black,top color=blue!30,bottom color=blue!50,middle color=blue!20,shading angle=0]
\tikzstyle{water}=[draw=mydarkblue,top color=watercol!90,bottom color=watercol!90!black,shading angle=5]
\tikzstyle{vertical water}=[water,
  top color=watercol!90!black!90,bottom color=watercol!90!black!90,middle color=watercol!80,shading angle=90]
\def\tick#1#2{\draw[thick] (#1)++(#2:0.1) --++ (#2-180:0.2)}

\def\barometreTorricelli#1{
%parametre 1 : hauteur en mm de mercure => mmHg
% PRESSURE TORRICELLI

  %\def\mmHg{760}
  \def\mmHg{#1}
  \def\Rx{1.8}
  \def\Ry{0.05}
  \def\rx{0.18}
  \def\ry{0.06*\rx/1.5}
  \def\H{1.0}
  \def\h{0.72*\H}   % water level height
  \def\th{3.2*\H}   % tube height
  \def\ty{\mmHg/1000*\th} % tube level
  \def\td{0.6*\h}   % tube depth
  \def\N{10}
  
  % WATER + CONTAINER
  \draw[vertical water] %rounded corners=2
    (-\Rx,\h) --++ (0,-\h) arc(180:360:{\Rx} and {\Ry}) --++ (0,\h);
  \draw[water]
    (0,\h) ellipse ({\Rx} and {\Ry});
  \draw[thick] (0,\H) ellipse ({\Rx} and {\Ry});
  
  % TUBE
  \draw[vertical water]
    (-\rx,\h) |-++ (2*\rx,\ty) --++ (0,-\ty);
  \draw[water]
    (0,\h+\ty) ellipse ({\rx} and {\ry});
  \draw[thick,line cap=round]
    (-\rx,\h) --++ (0,\th) coordinate (T) arc(180:0:\rx) --++ (0,-\th);
  \draw[mydarkblue,line cap=round]
    (-1.09*\rx,\h-0.005) arc(180:360:{1.09*\rx} and 1.12*\ry);
  \foreach \i [evaluate={\y=1.12*\H+0.84*\th*\i/\N}] in {0,...,\N}{
    \draw[line cap=round] (\rx,\y) arc(0:-50:{\rx} and \ry);
  }
  \begin{scope}
    \clip (-\Rx,\h) |-++ (2*\Rx,-\h) --++ (0,\h) arc(360:180:{\Rx} and {\Ry}) -- cycle;
    \draw[thick,line cap=round]
      (-\rx,\h) --++ (0,-\td) (\rx,\h) --++ (0,-\td);
      %(-\rx,\h) --++ (0,-\td) arc(180:360:{\rx} and {\ry}) --++ (0,\td);
    \draw[vertical water,opacity=0.5]
      (-\Rx,\h) --++ (0,-\h) arc(180:360:{\Rx} and {\Ry}) --++ (0,\h);
  \end{scope}
  \draw[mydarkblue]
    (-\Rx,\h) arc(180:360:{\Rx} and {\Ry});
  \draw[<->] (0.8*\Rx,\h) --++ (0,\ty) node[midway,fill=white,inner sep=1] {$h =\pgfmathparse{\mmHg}\pgfmathprintnumber{\pgfmathresult}~mm$};
  \draw[very thin,line cap=round]
    (T)++(130:\rx) node[anchor=-19,inner sep=1] {$Vide$} to[out=-50,in=170]++ (-30:1.5*\rx);
  \node[above] at (-0.6*\Rx,\H) {$P_\mathrm{atm}$};
  
  % CONTAINER
  \draw[thick]
    (-\Rx,\H) --++ (0,-\H) arc(180:360:{\Rx} and {\Ry})
              --++ (0,\H) arc(360:180:{\Rx} and {\Ry}) -- cycle;
              
  }
	


	
%\end{document}